\section{Bedeutung der Atom- und Molekülphysik} % (fold)
\label{sec:Bedeutung der Atom- und Molekülphysik}
\begin{erl}{Atomphysik}
    mikroskopischer Aufbau der Materie, d.h der Struktur der Atome un ihrer
    gegenseitugen Wechselwirkung. 
\end{erl}

\begin{erl}{Ziel}
    Eigenschaften der makroskopischen Materie aus ihrem mikroskopischen Aufbau
    zu verstehen
\end{erl}

Atom- und Molekülphysik bildet die Grundlage der
\begin{itemize}
    \item Thermodynamik (für statistische Beobachtungen)
    \item Atmosphährenphysik, Meteorologie (Wetter)
    \item Festkörperphysik
    \item Astrophysik (Absorption und Emission von Strahlung)
    \item Licht-Materie Wechselwirkung
    \item Laserlicht
\end{itemize}
Atom- und Molekülphysik bildet darüber hinaus die Grundlage der Chemie und
zunehmend der Biologie und Medizin:
\begin{itemize}
    \item Einordnung der Atome im Periodensystem
    \item Molkeülbildung, -bingungen, -struktur
    \item chemische Reaktionen (Dynamit)
    \item biologische Prozesse (Photosynthese, Energieproduktion in Zellen,
    Ionentransport durch Zellmembran, Nervenleitung)
\end{itemize}
$\rar$ Molekularbiologie + Molekularmedizin

Atomphysik spielt eine wichtige Rolle in der modernen Technik
\begin{itemize}
    \item Entwicklung des Lasers (Messtechnik, Nachrichtentechnik,
    Produktionstechnik Medizin)
    \item Messtechnik (Oszillograph, Spektrographen, Tomographen)
    \item Halbleitertechnik (integrierte Schaltung)
    \item Medizintechnik (Spurenelemente ???)
    \item Umwelttechnik
    \item Energietechnik (Solarzelle, alternative Antriebstechniekn wie z.B
    Brennstoffzelle)
\end{itemize}
\begin{erl}{Atomphysik}
Ausganspunkt für die Entweicklung der Quantemechanik und damit für unseres
heutiges physikalisches Weltbild (probabilistische Beschreibung der Physik,
Heisenbergsche Unschärferelation, Welle-Teilchen-Dualismus, nicht-lokal
verschränkte Zustände)
\end{erl}
% section Wiederholung klassische Mechanik (end)
