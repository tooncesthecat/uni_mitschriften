\section{Hinweise auf die Existenz der Atome} % (fold)
\label{sec:Hinweise_auf_die_Existenz_der_Atome}
\subsubsection{Daltons Gesetz der konstanten Proportionen} % (fold)
\label{ssub:Daltons_Gesetz_der_konstanten_Proportionen}
Durch genaues Wiegen der Massen von Reaktanden und Reaktionsprodukten vor und
nach einer chemischen Reaktion erkannte Dalton:

"Die Massenverhältnisse der Stoffe, aus denen sich eine chemische Verbindung
bildet ist für jede Verbindung konstant."

\begin{beis}
   $100g$ Wasser bilden sich immer aus $11.1g$ Wasserstoff und $88.9g$
   Sauerstoff $\rar$ Massenverhältnis $1\cl8$ ($H_2 O$ mit $\ _1^1 H$ und $\
   _8^{16} O$). 
\end{beis}

$\rar$ Atomhypothese (Dalton $1805$):

"Das Wesen chemischer Umwandlungen besteht in der Vereinigung oder Trennung von
Atomen".

\begin{erlit}{$3$ Postulate ($1808$):}
    \item alle elementaren Stoffe bestehen aus kleinsten Teilchen (Atomen), die
    man chemisch nicht weiter zerlegen kann
    \item alle Atome desselben Elements sind in Qualität, Größe und Masse
    gleich. Die Eigenschaften einer chemischen Substanz werden durch diejenigen
    seiner Atome bestimmt.
    \item wenn chemische Susbtanzen eine Verbindung eingehen, so vereinigen
    sich die Atome der beteiligten Elementen immer in ganzzahligen
    Massenverhältnissen
\end{erlit}
\begin{beis}
    $H$ und $O$ bilden $H_2 O$, aus Masseverhältnis
    $\frac{m(H)}{m(O)}=\frac{1}{16}$ erhält man das gemessen Massenverhältnis
    $\frac{m(2H)}{m(O)}= \frac{1}{8} = \frac{11.1g}{888g}$
\end{beis}
Dalton bezog alle Atommassen auf das $H$-Atom und nannte die relative Atommasse
$\frac{m_x}{m_H}$ eines Elementes $x$ das Atomgewicht (ohne Einheiten).

Heute wird $\frac{1}{12} \ ^{12}C$ statt $H$ als Bezugsmasse verwendet, genannt die
"atomare Masseneinheit (AME)", bzw englisch "$u$" (atomic mass unit)
\begin{equation*}
    1u=1AME=\frac{m\lk ^{12}C \rk}{12}=1.66055 \cd 10^{-17}kg
\end{equation*}
Daltons Atomgewicht wird "atomare bzw. molekulare Massenzahl $A$" genannt.
\begin{beis}
    Sauerstoff hat atomare Massenzahl $16$ und ein Gewicht von $16u$.
\end{beis}
\begin{align*}
    \text{$\ ^m$Luft}
    &= 0.75m_{N_2} + 0.25 m_{O_2} \\
    &= 0.75 \cd 28 u + 0.25 \cd 32 u \\
    &= 29u \\
    m_{He}
    &= 4u
\end{align*}
\begin{equation*}
    \frac{m_{\text{Luft}}}{m_{He}} = \frac{29u}{4u} = 7.25
\end{equation*}
% subsubsection Daltons Gesetz der konstanten Proportionen (end)
\subsection{Gay-Lussac's Gesetz} % (fold)
\label{sub:Gay-Lussac's_Gesetz}
$1805$ entdeckten J.L Gay-Lussac($1778$-$1850$) und A.v.Humboldt
($1769$-$1859$) unabhängig voneinander: gasförmiger Sauerstoff u. Wasserstoff
verbinden sich bei gleichem Druck u. Temperatur immer im Verhältnis von $1 \cl
2$ Raumteilchen.

Nach weiteren Experimenten formulierte Gay-Lussac:

\begin{quote}
"Vereinigen sich zwei oder mehr Gase restlos zu einer chemischen Verbindung, so
stehen ihre Volumina bei gleichem Druck und Temperatur in Verhältnis ganzer
Zahlen"
\end{quote}
\begin{beis}
    $2l$ $H_2$ und $1l$ $O_2$ ergeben $2l$ $H_2O$ (und nicht $3l$ $H_2O$).
\end{beis}
A.Avogadro($1776$-$1856$) erklärte diese Resultate durch Einführung des
Molekülbegriffs:
\begin{quote}
    "Ein Molekül ist das kleinste Teilchen eines Gases, das noch die chemischen
    Eigenschaften dieses Gases besitzt. Ein Molekül besteht aus zwei oder mehr Atomen."
\end{quote}
Mit Ergebnissen von Gay-Lussac stllte Avogadro die Hypothese auf:
\begin{quote}
    "Bei gleichem Druch und gleicher Temperatur enthalten verschiedene Gase bei
    gleiem Volumina die gleiche Zahl von Molekülen."
\end{quote}
$\rar$ Definition des $mol$:
\begin{quote}
    $1 mol$ eines Gases entspricht einer Anzahl von Molkülen deren Masse gleich
    der molekularen Massenzahl $A$ der Gasmoleküle in Gramm ist."
\end{quote}
In moderner Formulierung (bezogen auf $^{12}C$), die auch für nicht-gasförmige
Stoffe gilt: 
\begin{erl}{Mol}
    $1mol$ ist die Stoffmenge, die ebenso viele Teilchen (Atome oder Moleküle)
    enthalt wie $12g$ Kohlenstoff $^{12}C$
\end{erl}
\begin{beis}
    $1mol$ $H_2$ wiegt $2g$ \\
    $1mol$ $O_2$ wiegt $32g$ \\
    $1mol$ $H_2O$ wiegt $18g$ 
\end{beis}
Die Zahl $N_A$ der Moleküle in der Stoffmenge $1mol$ heißt
"Avogadro-Konstante". Ihr Wert ist
\begin{equation*}
    \boxed{
    N_A = 6.022141510 \cd 10^{23} mol^{-1}
    }
\end{equation*}
Aus der Avogadro-Hypothese folgt:

$1mol$ eines beliebigen Gases nimmt unter Normalbedingung ($p=1bar, \quad
T=0^{\circ}C$) immer das gleiche Volumen ein. Der experimentell bestimmte WErt
für das Molkvolumen ist:
\begin{equation*}
    \boxed{
        V_M=22,41399637 dm^3
    }
\end{equation*}
Das Molgewicht $M_x$ eines Stoffes $x$ ergibt sich daraus zu:
\begin{equation*}
    \boxed{
        M_x = N_a \cd m_x = N_A \cd A \cd u = A\text{(Gram)}
    }
\end{equation*}
% subsection Gay-Lussac's Gesetz (end)
\subsection{kinetische Gastheorie} % (fold)
\label{sub:kinetischer_Gastheorie}
Die von L. Boltzmann, R.Clausius und J.C.Maxwell im $19$. Jahrhundert
entwickelte kinetische Gastheorie erlaubt viele makroskopische Eigenschaften
eines Gases auf die Bewegung von Gasmolekülen und deren Wechselwirkung bei
Stößen zurückzuführen. Sie hat wesentlich zur Untermauerung der Atomhypothese
beigetragen:
\begin{itemize}
    \item Clausius leitete $1857$ die Zustandsgleichung des idealen Gases $pV=N
    k_B T$ under Annahme sich bewegender Moleküle der mittleren quadratischen
    Geschwindigkeit $\bar{v^2}$ ab: 
    \begin{equation*}
        pV = \frac{1}{3} m \bar{v^2} = N k_B T
    \end{equation*}
    \item Die Temperatur eines Gases lässt sich damit auf die mittlere
    kinetische Energie eines Gasmoleküls zurückführen:
    \begin{equation*}
        \left\langle E_{kin} \right\rangle
        =
        \frac{1}{2} m \bar{v^2} 
        =
        \frac{3}{2} k_B T
    \end{equation*}
    \item mit der inneren Energie $U$ eines Gases
    \begin{equation*}
        U = N \frac{f}{2} k_B T
    \end{equation*}
    ergibt sich die Wärmekapazität bei konstantem Volumen $C_V$ zu:
    \begin{equation*}
        C_V = N \frac{f}{2} k_B
    \end{equation*}
    mit $f$ der anzahl der Freiheitsgraden eines einzelnen Gasmoleküls.
    \item Transportprozesse in Gasen wie Teilchenstrom (Diffusion), Wärmestrom
    und Impulsstrom können auf die mittlere Freue Weglänge $\Lambda =
    \frac{1}{n \cd \sigma}$, die mittlere Geschwindigkeit $\bar{v}$ und $f$
    zurückgeführt werden.
\end{itemize}

% subsection kinetische Gastheorie (end)
% section Hinweise auf die Existenz der Atome (end)
