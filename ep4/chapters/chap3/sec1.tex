\section{Bestimmung der Atomgröße} % (fold)
\label{sec:Bestimmung_der_Atomgroesse}
Verschiedene experimentelle Verfahren erlauben es, die Atomgröße abzuschätzen

\subsection{Bestimmung aus dem Kovulemen der v.d. Waals Gleichung} % (fold)
\label{sub:Bestimmung_aus_dem_Kovulemen_der_v.d._Waals_Gleichung}
Ein reales Gas wird in guter Nährung durch die van-der-Waals'sche
Zustandsgleichung beschrieben:
\begin{equation*}
    \lk p + \frac{a}{V_M^2}\rk \lk V_M -b \rk = R \cd T \quad \text{Für Molzahl
    $n=1$}
\end{equation*}
\begin{itemize}
    \item $a$: Binnendruck-Konstante
    \item $b=4 N_A V_a$ gibt das Vierfache des Eigenvolumens aller $N_A$ Atome
    im Molvolumen $V_M$ an.
\end{itemize}
Durch Bestimmung von $b$ aus der Messung $p(T)$ bei konstantem Gasvolumen $V_M$
ergibt sich:
\begin{equation*}
    \boxed{
        V_a = \frac{b}{4 \cd N_A}
        }
\end{equation*}
% subsection Bestimmung aus dem Kovulemen der v.d. Waals Gleichung (end)

\subsection{Abschätzung aus dem Transportkoeffizienten in Gasen} % (fold)
\label{sub:Abschätzung_aus_dem_Transportkoeffizienten_in_Gasen}
Wie oben gezeigt, hängen die Transportkoeffizienten in Gasen für den
Teilchenstrom $j_x$ (Diffusionskonstande $D$), Wärmestrom $Q$
(Wärmeleitfähigkeit $\lambda$) und Impulsstrom $j_p$ (Viskosität $\eta$) von
der mittleren freuen Weglänge ab.
Dabei ist
\begin{equation*}
    \Lambda = \frac{1}{n \cd \sigma}=\frac{k_B T}{p \cd \sigma}
\end{equation*}
mit $\sigma$ dem Sotßqueerschnitt der Atome. Im Modell starrer Kugeln ist
\begin{equation*}
    \sigma = \pi \lk 2r \rk^2
\end{equation*}
(allg: $\sigma = \pi (r_1 + r_2)^2$)

Man kann also durch Messung der Transportkoeffizienten und damit von $\Lambda$
Information über $\sigma$ und damit über $r$ erhalten:
\begin{beis}
    \begin{equation*}
        D = \frac{1}{3} \bar{v} \Lambda 
        = \frac{1}{3p} \sqrt{\frac{\lk 2 k_B T \rk)^3}{\pi m} \frac{1}{\sigma}}
    \end{equation*}
    mit $\bar{v} = \sqrt{\frac{8 k_B T}{\pi m}}l$ und $n= \frac{p}{k_B T}$
\end{beis}
% subsection Abschätzung aus dem Transportkoeffizienten in Gasen (end)

\subsection{Beugung von Röntgenstrahlen an Kristallen} % (fold)
\label{sub:Beugung_von_Röntgenstrahlen_an_Kristallen}
Interferenz von Röntgenstrahlen an den Netzebenen eines Kristalls kann zur
Bestimmung von Atomvolumina verwendet werden. Aus dem Abstand $d$ definierter
Netzebenen in einem Kristall lässt sich das Volumen der Einheitszelle $V_E$ des
Kristalls bestimmen. Kennt man den Raumfüllungsfaktor $f = \frac{\sum
V_a}{V_e}$ der Atome, so ergibt sich bei $N_E$ Atomen pro einheitszelle 
\begin{equation*}
    \boxed{
        V_a = f \frac{V_E}{N_E}
    }
\end{equation*}
% subsection Beugung von Röntgenstrahlen and Kristallen (end)
\subsection{Lenard-Jones Potenzial} % (fold)
\label{sub:Lenard-Jones_Potenzial}
Die Methoden $1$ - $3$ geben zwar die gleiche Größenordnung, aber dennoch
verschiedene Werte für die Atomradien.

Die Unterschiede resultieren aus der unklaren Definition des Atomradius. Bei
einer starren Kugel ist Radius $r_0$ wohldefiniert. Die realen Atome erzeugen
einen Stoßquerschnitt jedoch aus einem Wechselwirkungspotenzial, dessen Verlauf
recht gut durch das Lennard-Jones-Potenzial beschrieben wird:
\begin{equation*}
    V(r)= \frac{a}{r^{12}} - \frac{b}{r^6}
\end{equation*}
wobei Konstaten $a$ und $b$ von der Atomsorte abhängig sind.

Der Atomradius kann nun durch das Minimum bei
\begin{equation*}
    r_m = \lk \frac{2a}{b} \rk^{\frac{1}{6}}
\end{equation*}
oder dem Nulldurchgang bei
\begin{equation*}
    r_0 = \lk \frac{a}{b} \rk^{\frac{1}{6}}
\end{equation*}
definiert werden. Generell haben die Radien von Atomen die Größenordnung $r
\sim 10^{-10}a m = 1 \angstrom$.

Zur genauen Beschreibung der Aromgröße gibt man den Verlauf $V(r)$ an, Diesen
erhält man aus Messung der Ablenkung von Atomen bei Stoßprozessen oder aus
Spektroskopie von Molekülen.
% subsection Lenard-Jones Potenzial (end)


% subsection Bestimmung der Atomgröße (end)

% section Masse, Größe und Struktur der Arome (end)
