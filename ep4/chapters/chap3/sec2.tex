\section{Bestimmung der Atommasse} % (fold)
\label{sec:Bestimmung_der_Atommasse}
\subsection{einfache Verfahren} % (fold)
\label{sub:einfache_Verfahren}
Die einfachsten Verfahren zur Bestimmung der Atommasse verwenden die Abogadro
Konstanta $N_A$ und die atomare Massenzahl $A$ oder wiederum Beugung von
Röntgenstrahlen an Kristallen.
\begin{itemize}
    \item Misst man die Masse $M_M$ eines Molvolumens eines atomaren Gases
    unter Normalbedingungen ($p=1bar, \quad T=0^{\circ}, \quad \rar V_M =
    22,4l$), so ist die Masse eines Atoms
    \begin{equation*}
        m_x = \frac{M_M}{N_A}
    \end{equation*}
    \item In einem Kristall kann man die Atomabstände mit Hilfe von
    Röntgenbeugung messen. Aus den Abmessungen des Gesamtkristalls erhält man
    die Gesamtzahl $N$ der in dem Kristall befindlichen befindlichen Atomen.
    Bei diner Gesamtmasse $M_k$ des Kristall ergibt sich
    \begin{equation*}
        m_x = \frac{M_k}{N}
    \end{equation*}
\end{itemize}
% subsection einfache Verfahren (end)
\subsection{Massenspektrometer} % (fold)
\label{sub:Massenspektrometer}
Die genauste Methode der Massenbestimmung von Atomen benutzt die Ablenkung von
Ionen in elektrischen oder magnetischen Feldern.
Aus der gemessenen Masse $m(A^+)$ eines einfach geladenen Ions $A^+$ erhält man
die atommasse durch:
\begin{equation*}
    m(A) = m(A^+) + m(e^-) - \frac{1}{c^2}E_B
\end{equation*}
mit $E_B$ der Bindungsenergie des Elektrons im Atom $A$.
\subsubsection{Massenspektrograph nach Thomson} % (fold)
\label{ssub:Massenspektrograph_nach_Thomson}
%BILD
Die durch die Gasentladung erzeugten Ionen (Masse $M$, Ladung $q$) wrden zur
Kathode hin bescheleunigt, auf Geschwindigkeit 
\begin{equation*}
    v=\sqrt{\frac{2qU}{m}} \quad \lk E_{kin} = \frac{1}{2} m v_t^2 = qU \rk 
\end{equation*}
Im homogenen Magnetfeld $\vec{B}=$ %(B 0 0)$
und homogenen elektr Feld $\vec{E}=$ %(E 0 0)$
erfahren die Ionen über die Länge $L$ eine Ablenkung    
\begin{equation*}
    \label{eqn:massenspek1}
    \tag*{\text{(*)}}
    \frac{d ^2 x}{d t^2} = \frac{1}{m}E
\end{equation*}
\begin{equation*}
    \label{eqn:massenspek2}
    \tag*{\text{(**)}}
    \frac{d ^2y}{d t^2}=\frac{q}{m} v B
\end{equation*}
Mit $v_x,v_y \ll v_z \simeq v$ hat man:
\begin{gather*}
    \frac{dx}{dt} = \frac{d x}{d t} \frac{d z}{d t} \simeq \frac{d x}{d z} \\
    \rar \frac{d ^2x}{d t^2} \simeq v^2 \frac{d ^x}{d z^2}
\end{gather*}
und ebenso:
\begin{equation*}
    \frac{d ^2y}{d t^2} \simeq v^2 \frac{d ^2y}{d z^2}
\end{equation*}
Einsetzen in $\lk * \rk$ ergibt:
\begin{equation*}
    \frac{d ^2 x}{d z^2} =\frac{q}{m} \frac{1}{v^2}E
\end{equation*}
und
\begin{equation*}
    \frac{d ^2y}{d z^2} = \frac{q}{m}\frac{1}{v}B
\end{equation*}
Für $z \in \left[ -\frac{L}{2}, + \frac{L}{2} \right]$ liefert die Integration
\begin{align*}
    \frac{d x}{d z} &= \int_{-\frac{L}{2}}^{z} \frac{qE}{mv^2}dz' =
    \frac{qE}{mv^2} \lk \frac{L}{2} +z \rk  \\
    \rar x(z) &= \frac{qE}{wmv^2}\lk \frac{L}{2} +z \rk^2
\end{align*}
Ebenso ergibt sich mit $\lk ** \rk$
\begin{equation*}
    y(z) = \frac{qB}{2mv} \lk \frac{L}{2} + z \rk^2
\end{equation*}
Am Ende des Kondensators befinted sich das Ion bei $\vec{r}= \lk
x(\frac{L}{2}),y(\frac{L}{2}),\frac{L}{2}\rk$. Die Steigung in $x$-Richung ist
dort:
\begin{equation*}
     \frac{dx }{dz }\lk \frac{L}{2} \rk = \frac{qE}{mv^2} \cd L
\end{equation*}
Bei der Fotoplatte bei $z=z_0$ ist die $x$-Koordinate also:
\begin{align*}
    x(z_0) &= \frac{qE}{2mv^2} \cd L^2 + \frac{qEL}{mv^2} \lk z_0 -
    \frac{L}{2}\rk \\
    &= \frac{qEL}{mv^2}\cd z_0
\end{align*}
analog erhalt man für die $y$-Koordinate:
\begin{equation*}
    y(z_0) = \frac{qB}{2mv}L^2 + \frac{qBL}{mv}\lk z_0 - \frac{L}{2}\rk =
    \frac{qBL}{mv} \cd z_0
\end{equation*}
Eliminiation von $v$ liefert bei $z_0$:
\begin{equation*}
    x(y) = \frac{m}{q}\underbrace{\frac{E}{B^2 L z_0}}_{=f(\frac{q}{m}) \cd
    y^2}\cd y^2
\end{equation*}
Für jeden Wert $\frac{q}{m}$ entspricht dies einer Parabel, aus deren Vorfaktor
$f(\frac{q}{m})$ bei bekannten $E$ und $B$ das Verhältnis $\frac{q}{m}$
bestimmt werden kann.

Beim Thomson-Massenspektrograph werden gleiche $\frac{q}{m}$ mit verschiedenen
Geschwindigkeiten auf verschiedene Stellen der Fotoplatte entlang der Parabel
abgebildet. Ein besseres Signal erhält man, wenn die Ionen mit gleichen
$\frac{q}{m}$-auch bei unterschiedlichen Geschwindigkeiten - auf einen
einzelnen Punkt fokusiert sind. 
% subsection Massenspektrometer (end)

\subsection{Astonsche Massenspektrograph} % (fold)
\label{sub:Astonsche_Massenspektrograph}
$\rar$ ÜA
% subsection Astonsche Massenspektrograph (end)
\subsection{Flugzeit-Massenspektrometer} % (fold)
\label{sub:Flugzeit-Massenspektrometer}
Ein einfaches, billiges und auch heute noch weit verbreitetes
Massenspektrometer ist das Flugzeit-Massenspektrometer. Dabei werden IOnen in
einem kleinen Volumen erzeugt (z.B durch Photoionisation mit einem geputztem
Laser), durch eine Spannung $U$ auf $v = \sqrt{\frac{2qU}{m}}$ beschleunigt und
durchlaufen eine feldfreie Strecke $L$, bevor sie vom Detektor registriert
werden (Chameltron, Ionenmultiplier, Kanalplattenverstärker) registriert
werden.
%BILD
Die Zeit zwischen Erzeugung und Detektion
\begin{gather*}
    T_m = \frac{L}{v} = \frac{L}{\sqrt{\frac{2qU}{m}}} \\
    \rar
    \boxed{
        m = \frac{2qU}{L^2}T_m
    }
\end{gather*}
Die Genauigkeit des Massenspektrometers hängt davon ab, wie genau $U$, $L$ und
$T$ bestimmt werden können, d.h u.a in welcher Zeit $\Delta t$ und in einem wie
kleinem Volumen $\Delta V$ sie erzeugt werden können.
Die genaueste Auflösung wird heutzutage mit sogenanten
Ionen-Zykloyton-Resonanz-spektrometern (sog. Penning-Fallen) erzielt. Die
erreichte Auflösung ist $\frac{\Delta m}{m} \leq 10^{-8}$.
% subsection Flugzeit-Massenspektrometer (end)

\subsection{Isotopie} % (fold)
\label{sub:Isotopie}
Durch genaue Massenmessungen wurde es möglich nachzuweisen, dass viele
chemische Elemente in der Natur in verschiedenen Isotopen vorkommen. Die
Isotope eines Elements haben alle die gleichen chemischen Eigenschaften,
unterscheiden sich aber in ihrer Masse
um kleine ganzzahlige Vielfache einer $AME$. Die Erklärung der Isotopie kam
erst mit der Entdeckung des Neutrons: Isotope unterscheiden sich in der anzahl
der Neutronen im Atomkern. Ein Element (wie z.B $He$) wird vollständig erst
durch die Anzahl der Elektronen (= Protonen = Ladungszahl $Z$) und Anzahl der
Neutronen und Protonen im Kern (= atomare Massenzahl $A$) angegeben.
\begin{beis}
    Helium mit $2$ Protonen und $1$ Neutron $= ^3_2 He$.

    Helium mit $2$ Protonen und $2$ Neutronen $= ^4_2 He$.
\end{beis}
% subsection Isotopie (end)

% section Bestimmung der Atommasse (end)
