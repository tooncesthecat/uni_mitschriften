\section{Bestimmung der Atommasse} % (fold)
\label{sec:Bestimmung_der_Atommasse}
\subsection{einfache Verfahren} % (fold)
\label{sub:einfache_Verfahren}
Die einfachsten Verfahren zur Bestimmung der Atommasse verwenden die Abogadro
Konstanta $N_A$ und die atomare Massenzahl $A$ oder wiederum Beugung von
Röntgenstrahlen an Kristallen.
\begin{itemize}
    \item Misst man die Masse $M_M$ eines Molvolumens eines atomaren Gases
    unter Normalbedingungen ($p=1bar, \quad T=0^{\circ}, \quad \rar V_M =
    22,4l$), so ist die Masse eines Atoms
    \begin{equation*}
        m_x = \frac{M_M}{N_A}
    \end{equation*}
    \item In einem Kristall kann man die Atomabstände mit Hilfe von
    Röntgenbeugung messen. Aus den Abmessungen des Gesamtkristalls erhält man
    die Gesamtzahl $N$ der in dem Kristall befindlichen befindlichen Atomen.
    Bei diner Gesamtmasse $M_k$ des Kristall ergibt sich
    \begin{equation*}
        m_x = \frac{M_k}{N}
    \end{equation*}
\end{itemize}
Die genauste Methode der Massenbestimmung von Atomen beuntzt die ablenkung von
Ionen in elektrischen oder magnetischen Feldern.
Aus der gemessenen Masse $m(A^+)$ eines einfach geladenen Ions $A^+$ erhält man
die atommasse durch:
\begin{equation*}
    m(A) = m(A^+) + m(e^-) j- \frac{1}{c^2}E_B
\end{equation*}
mit $E_B$ der Bindungsenergie des Elektrons im Atom $A$.

% subsection einfache Verfahren (end)

% section Bestimmung der Atommasse (end)
