\section{Innere Struktur der Atome} % (fold)
\label{sec:Innere_Struktur_der_Atome}

Bis zum Ende des $19.$ Jahrhunderts galt als gesichert, dass Materie aus
elektrisch geladenen Teilchen aufgebaut ist.
Hinweise darauf:
\begin{enumerate}
    \item Elektrolyse: Moleküle dissoziieren im elektreoschen Feld in positive
    und negative Ladungsträger (Ionen)
    \item Magnetfeldeffekte bei elektrische Leitung in Metallen (Hall-Effekt)
    \item Ablenkung von $\alpha$- und $\beta$-Strahlung im Magnetfeld bei
    radioaktivem Zerfall
    \item Experimente mit Gasentladung: erzeugte "Kanal- und Kathodenstrahlen"
    werden durch elektrische und magnetische Felder beeinflusst.
\end{enumerate}
\subsection{Kathoden- und Kanalstrahlen} % (fold)
\label{sub:Kathoden-_und_Kanalstrahlen}
Durch Ablenkung von Kathodenstrahlen im $E$- bzw. $B$-Feld konnte J.J.Thomson
$1897$ das Verhältnis $\frac{q}{m} = \frac{e}{m}$ bestimmen. Er stellte fest,
dass dieses Verhältnis unabhangig vom Kathodenmaterial ist, und ca $10^4$ mal
größer als bei den von $1886$ von E. Goldstein ($1850$-$1930$) entdeckten
Kanalstrahlen. Bereits $1895$ werden diese Elementarladungen von G. Stoney
"Elektronen" genannt.
W.Wien ($1864$-$1928$) konnte $1897$ den WERt $\frac{q}{m}$ der Kanalstrahlen
messen und zeigen, dass sie aus den positiven Ionen des Füllgases der
Entladungsröhre bestehen. In Kombination mit der Atomvorstellung ergab sich
daraus:

Atome sind aus elektrisch geladenen Teilchen aufgebaut. $\rar$ es gibt eine
Substruktur!

Da Atome neutral sind, müssen gleich viele negative wie positive Lasungen
vorhanden sein, d.h:
Atome bestehen aus negativ geladenen Elektronen und einer
entgegengesetzt gleichen positiven Ladung!
Die elektrisch geladenen Bausteine haben Ladung und Masse, wobei
$\frac{m_e}{m_+} < 10^{-3}$.

Daraus ergeben sich folgende Fragen:
\begin{enumerate}
    \item Welche Eigenschaften haben die Substrukturteilchen?
    \item Wie werden die Telchen im Atom zusammengehalten?
    \item Wie sind sie im Atom angeordnet, d.h. welche Substruktur hat das
    Atom?
\end{enumerate}
% subsection Kathoden- und Kanalstrahlen (end)
\subsection{Differentielle Wirkungsquerschnitt bzw. Streuungsquerschnitt} % (fold)
\label{sub:Differentielle_Wirkungsquerschnitt_bzw._Streuungsquerschnitt}
Um die Verteilung der Ladung innerhalb der Atome zu bestimmen, kann man
elektrisch geladene Teilchen mit Ladung $q_s$ als Sonde verwenden. Deren
Ablenkung im Coulombpotenzial des Atoms
\begin{equation*}
    V_c (r) = \frac{1}{4\pi\epsilon}\frac{q_s \cd q_A}{r} 
    \quad
    \text{mit} 
    \quad
    q_A = \int \rho_A (\vec{r}) dV
\end{equation*}
erlaubt Rückschlüsse über die Ladungsverteilung $\rho_A (\vec{r})$ zu ziehen.
Dazu werden die "Sonden-Teilchen" in einem Winkelbereich $\left[ \vartheta, \vartheta
+ d \vartheta \right]$ um die Probe nachgewiesen. Diese gehören - abhängig vom
Streupotenzial $V(r)$ - zu einem "Stoßparameter" $b$ im Bereich $\left[ b, b+db
\right]$. 
%BILD
Der "differentielle Wirkungsquerschnitt" $d \sigma$ gibt hierbei den Bruchteil
des gesamten (integralen) Streuquerschnitts $\sigma$ and, durch den die
Teilchen hindurchfliegen, die vom Potenzial $V(r)$ in dem vom Detektor
aufgespannten Raumwinkel $d \Omega = \sin \vartheta d \vartheta d\phi$
gestreut werden. D.h.:
\begin{gather*}
    \frac{d \sigma (\vartheta,\Omega)}{\sigma}
    =
    \frac{\Delta \dot{N}(\vartheta,\Omega)}{\dot{N}}
    =
    \frac{2 \pi b db}{\sigma} \frac{\varphi}{2\pi}
    =
    \frac{2\pi}{\sigma} \frac{db}{d\vartheta}d\vartheta \frac{d \varphi}{2 \pi}
    =
    \frac{b}{\sigma} \frac{db}{d\vartheta}d\vartheta d\varphi \\
    \rar d\sigma(\vartheta,\Omega) = b \frac{d b}{d \vartheta} \frac{1}{\sin
    \vartheta} d\Omega \\
    \rar
    \boxed{
    \frac{d \sigma (\vartheta,\Omega)}{d \Omega} = b \frac{d b}{d \vartheta}
    \frac{1}{\sin \vartheta}
    }
\end{gather*}
% subsection Differentielle Wirkungsquerschnitt bzw. Streuungsquerschnitt (end)
\subsection{Das Thomsonsche Atommodell} % (fold)
\label{sub:Das_Thomsonsche_Atommodell}
J.J. Thomson ($1856$-$1940$) schlug $1903$ für die raüliche Verteilung der $Z$
Elektronen und $Z$ positiven Ladungen das sog "Rosinenkuchen-Modell", bei dem
die positiven Ladungen gleichäßig über die gesamte Fläche verteilt sind und die
$e^-$ sich frei darin bewegen.
\begin{equation*}
    n_1 = n_+ = \frac{Z}{4 \frac{\pi}{3} R^3} \quad \text{$R$: Atomradius}
\end{equation*}
Würden nur die positiven Ladungen berücksichtigt, würde ein Elektron im Abstand
$r<R$ bei einer homogenen Ladungsverteilung (mit Ladungsdichte $= \rho_+ =
\frac{Ze}{\frac{4 \pi}{3}R^3}$) das Feld
\begin{equation*}
    E(\vec{r}) = \frac{Zer}{4\pi\epsilon R^3} \h{r}
\end{equation*}
und somit die Kraft
\begin{equation*}
    \vec{F}(\vec{r}) 
    =
    -e \vec{E}
    = - k_T \vec{r} \quad \text{mit $k_T = \frac{Ze^2}{4\pi\epsilon R^3} \h{r}$}
\end{equation*}
wirken. Das führt zu einer harmonsichen Schwingung mit Frequenz $\omega_e =
\sqrt{\frac{k_T}{m_w}}$. Für eine homogene Verteilung aller $Z$ Elektronen
(Teilchendichte $n_e = \frac{Z}{\frac{4\pi}{3} R^3}$) ergibt sich eine
kollektibe Schwungung gegen die viel schwereren positiven Ladungen mit der
sog. Plasmafrequenz.
\begin{align*}
    w_p 
    &=
    \sqrt{ \frac{n_e \cd e^2}{\epsilon \cd m_e}}
    = 
    \sqrt{\frac{3 Z e^2}{4 \pi \epsilon_0 m_e R^3}} \\
    &=
    \sqrt{3} \omega_e
\end{align*}
\begin{erl}{Problem:}
    Die aus diesem Modell abgeschätzten Absorptions und Emissionsfrequenzen
    stimmen nicht mit den beobachteten atomaren Frequenzen überein. Vorallem:
    Ergebnisse von Streuezperimenten stimmen nicht mit der vom Modell
    erwarteten Winkelverteilung überein:
\end{erl}
%\subsubsection{Winkelverteilung von $\alpha$-Teilchen} % (fold)
%\label{ssub:Winkelverteilung_von_$\alpha$_Teilchen}
Für $\alpha$-Teilchen (Masse $m_{He}$, Ladung $2e$) spielen die Elektronen
wegen ihrer kleinen Masse bei der Ablenkung praktisch keine Rolle. Die Kraft
auf ein $\alpha$-Teilchen in Richtung senkrecht zu seiner Einfallsrichtung
ergibt sich daher für $r<R$ zu:
\begin{equation*}
    F_y = F(r) \sin \varphi
    \simeq
    \frac{2 Z e^2 r}{4\pi \epsilon_0 R^3} \frac{b}{r} 
    =
    2k_T b
\end{equation*}
%BILD
\begin{align*}
    \rar \Delta p_y 
    &=
    \int F_y dt = 2 k_T b T \\
    &\simeq \frac{4 k_T b }{\vartheta_0} \sqrt{R^2 -b^2}
    \quad \text{ mit $T \simeq \frac{d}{\vartheta_0}= \frac{2 \sqrt{R^2 -
    b^2}}{\vartheta_0}$}
\end{align*}
%BILD
Da $\Delta p_y \ll p_z \simeq p(dh \vartheta \ll 1)$
\begin{align*}
    \rar \frac{\Delta p_y}{p_z} 
    &\simeq \frac{\Delta p_y}{p} = \tan \varphi \\
    &=
    \frac{4 k_T b}{m \vartheta_0^2} \sqrt{R^2 - b^2} \simeq \vartheta
\end{align*}
\begin{equation*}
    \rar \boxed{
    \vartheta = \frac{4 k_T b}{m \vartheta_0^2} \sqrt{R^2 - b^2}
    } = \vartheta(b)
\end{equation*}
Für ein mit beliebigem Stoßparameter $b$ einlaufendes $\alpha$-Teilchen ergibt
sich der durchschnittliche Streuwinkel $\bar{\vartheta}$ zu:
%\begin{align*}
%    \bar{\vartheta}
%    &=
%    \int_{b=0}^R
%    \vartheta(b) \frac{2 \pi b}{\pi R^2} db
%    =
%    \frac{8 k_T}{m \vartheta_0^2 R^2}
%    \underbrace{\int_0^R \sqrt{R^2 -b^2}b^2 db}_{\frac{\pi}{16 R^4} \\
%    &=
%    \frac{\pi}{2} \frac{k_T R^2}{m \vartheta_0^2}
%    =
%    \frac{Z e^2}{8 \epsilon R m v_0^2}
%\end{align*}
\begin{beis}
    $R=0.1mm$, $\alpha$-Teilchen mit $E=5Mev$, and Goldfolie $(Z=79)$ gestreut
    werden
    \begin{equation*}
        \rar \bar{\vartheta} = 1.8 \cd 10^{-4} rad \simeq 0.62' \quad
        \rar \text{sehr kleiner Winkel}
    \end{equation*}
\end{beis}
Bisher Ablenkung durch ein Atom im Mittel berechnet ($\bar{\varphi}$). Im Experiment wurden die
$\alpha$-Teilchen durch eine Goldfolie geschossen und somit an vielen Atomen
gestreut (für Foliendicke von $10\mu m$ und Golddurchmesser von $\simeq 0.2 nm$, ca $6 \cd
10^4$ Atome).
Da die Stoßrichtung bei jeder Streuung  statistisch verteil ist (entweder nach
links oder rechts) is auch $\bar{\vartheta}$ statistisch verteilt. $\rar$
"Random walk"-Problem $\rar$ Gauß-Verteilung.

Die Zahl der in dem Winkel $\vartheta$ abgelenkten $\alpha$-Teilchen ergibt
such daher zu
\begin{equation*}
    N(\vartheta) = N_0 e^{- \frac{\vartheta^2}{\left\langle \Delta \vartheta
    \right\rangle^2}} = N_0 e^{- \frac{\vartheta^2}{m\bar{\vartheta}}}
\end{equation*}
Nach $m$ Stoßprozessen ergibt sich eine mittlere Breite der Verteilung
\begin{equation*}
    \tag*{\text{(Standardabweichung)}}
    \Delta \vartheta = \sqrt{m} \cd \bar{\vartheta}
\end{equation*}
\begin{beis}
    Goldfolie mit Dicke $10 \mu m, \quad m \simeq 5 \cd 10^4$.
    \begin{equation*}
        \rar \left\langle \Delta \vartheta \right\rangle
        =
        4 \cd 10^{-2} rad 
        \simeq 
        2.3"
    \end{equation*}
\end{beis}
Um das Thomsonsche Atommodell zu testen, führte Rutherford $1909$
Streuexperimente von $\alpha$-Teilchen an einer Goldfolie durch. Die Ergebnisse
zeigten, dass auch noch sehr große Streuwinkel (bis zu $160^{\circ}$) beobachtet
wurden. Dies widersprach eindeutig dem Thomson-Modell.

$\rar$ Alternativ schlug Rutherford ein anderes Modell fur Atomaufbau vor.
%subsection Das Thomsonsche Atommodell (end)
\subsection{Das Rutherfordsche Atommodell} % (fold)
\label{sub:Das_Rutherfordsche_Atommodell}
Rutherford schlug vor, dass die positive Ladung des Atoms in einem kleinem
volumen im Zentrum des Atoms, dem Atomkern, komprimiert ist. Dort ist auch die
nahezu gesamte Atommasse vereinigt. In dem Fall: Streuung der $\alpha$-Teilchen
im coulombfeld einer Punktladung $Q = Ze$
\begin{gather*}
    \rar \bar{E} (\vec{r}) = \frac{Ze}{4\pi \epsilon_0 r^2} \h{r} \\
    \rar \vec{F}(\vec{r}) = \frac{2Ze^2}{4 \pi \epsilon r^2} \h{r}
    = 2 \underbrace{k_R (r)}_{\neq const} \cd \vec{r}
\end{gather*}
Auch hier ist wieder:
\begin{equation*}
    F_y = F(r) \cd \sin \phi
\end{equation*}
%BILD
Wegen der Drehimpulserhaltung:
\begin{align*}
    \lv \vec{r} \times \vec{p} \rv
    =
    m v_0 \cd 
    =
    I \cd \omega
    &=
    m r^2 \cd \dot{\phi}
\end{align*}
\begin{equation*}
    \rar \frac{1}{r^2} = \frac{\dot{\phi}}v_0 \cd b{} \\
\end{equation*}
\begin{align*}
    \rar \Delta p_y
    &= \int_{-\infty}^{\infty} F_y dt
    = \overbrace{\frac{2 Ze^2}{4 \pi \epsilon_0}}^{M} \frac{1}{v_0 \cd b}
    \int_{-\infty}^{\infty} \sin \phi \frac{d \phi}{d t} dt \\
    \rar \Delta p_y 
    &=
    p \cd sin \vartheta \\
    &=
    M \cd \frac{1}{v_0 \cd b} \int_0^{\pi \cd \vartheta} \sin \phi d\phi \\
    &= 
    M \frac{1}{v_0 \cd b} \lk 1 + \cos \vartheta \rk 
\end{align*}
und $ \frac{1+\cos \vartheta}{\sin \vartheta} = \cot \frac{\vartheta}{2}$
ergibt sich die gesuchte Beziehung zwischen $b$ und $\vartheta$:
\begin{align*}
    b(\vartheta) 
    &=
    \frac{1}{p \cd \sin \vartheta} \cd \frac{M}{v_0} \lk 1 + \cos \vartheta \rk\\
    &=
    \frac{M}{m v_0^2} \cot \lk \frac{\vartheta}{2} \lk
\end{align*}
Mit $\frac{d \cot(x)}{d x } = - \frac{1}{\sin^2 (x)}$ ergibt sich daraus:
\begin{equation*}
    \frac{d b}{d \vartheta} = \frac{M}{2m v_0^2} \frac{1}{\sin^2 \lk
    fr{\vartheta}{2}}
\end{equation*}
Mit $\sin \vartheta = 2 \sin \frac{\vartheta}{2} \cos \frac{\vartheta}{2}$ kann
der differentielle Wirkungsquerschnitt $\frac{d \sigma}{d \Ohm}$ also
geschrieben werden als
\begin{equation*}
    \frac{d \sigma}{d \Ohm} 
    &=
    b \frac{d b}{d \vartheta} \cd \frac{1}{\sin \vartheta} \\
    &=
    \lk \frac{M}{m v_o^2}
    \frac{\cos(\frac{\vartheta}{2})}{\sin(\frac{\vartheta}{2})} \rk 
    \lk \frac{M}{2mv_0^2} \frac{1}{sin^2(\frac{\vartheta^2}{2})} \rk\\
    &=
    \lk \frac{1}{2\sin(\frac{\vartheta}{2})\cos(\frac{\vartheta}{2})} \rk
\end{equation*}
\begin{align*}
    \frac{d \sigma}{d \Omega}
    &= \frac{M}{(2mv_0^2)^2} \frac{1}{\sin^4 (\frac{\vartheta}{2})} \\
    &= \lk \frac{\frac{2e^2}{4\pi\epsilon}}{mv_0^2} \rk^2 \lk \frac{1}{\sin^4
    (\frac{\vartheta}{2})} \rk
\end{align*}
Die gemessen Streuverteilung stimmt sehr gut mit deisem Annahmen überein.
\begin{bem}
    \begin{itemize}
        \item  Bei $\alpha$-Teilchen mit $E_{kin} = 5Mev$ treten erst bei $\vartheta =
        150^{\circ}$ Abweichungen von der Rutherfordschen Streuformel auf.
        Dies entspricht eines Stoßparamters von $b= 6 \cd 10^{-15}m$, d.h. erst bei
        diesen Abständen beginnen andere Kräfte als die Coulombkraft wirksam zu
        werden $\rar$ Kernkräfte! 
        $\rar$ Kernradius $r_K \sim 10^{-15} m$
        \item Aus Experimenten mit verschiedenen Folienmaterialen konnte
        Chodwich $1920$ ableiten, dass $Z$ gleich der Ordnungzahl im
        periodischen System ist und dass
        \begin{equation*}
            r_K \simeq \lk 1.3 \cd 10^{-15} m \rk A^{\frac{1}{3}}
        \end{equation*}
        $\rar$ Rutherfordsches Atommodell:
        \begin{quote}
            Neg. Elektronen umkreisen den positiven Kern mit Kernladungszahl
            $Z$ (ohne Bewegung wäre keine Stabilität möglich.) Der Kern
            vereinigt nahzu die gesamte Masse des Atoms, macht aber nur den
            Bruchteil $\lk \frac{r_K}{r_A} \rk^3 \simeq 10^{-15}$ des
            Atomvolumens aus
        \end{quote}
    \end{itemize}
\end{bem}

% subsubsection Das Rutherfordsche Atommodell (end)

% subsubsection Winkelverteilung von $\alpha$ Teilchen (end)

% section Innere Struktur der Atome (end)
