\section{Kann man Atome sehen?} % (fold)
\label{sec:Kann man Atome sehen}
Auch wenn Atome ($r=0.1 - 0.3 nm$) mit sichtbarem Licht ($\lambda = 500nm$)
nicht aufgelöst werden können, so sind eine reihe von Techniken und
Instrumenten entwickelt worden, die eine indirekte oder direkte Beobachtung von
Atomen erlauben.
\subsection{indirekte Beobachtungen} % (fold)
\label{sub:indirekte_Beobachtungen}
\begin{enumerate}
    \item Brownsche Bewegung
    \item Auch die von C. Wilson $1911$ entwickelte Nebelkammer erlaubt es,
    indirekt die Bahn einzelner mikroskopischer Teilchen wie Atome, Ionen,
    Elektronen, Positornen etc. sichtbar zu machen: Ein Teilchen mit genügend
    hoher kinetischer Energiea kann hierbei durch Stöse mit Atomen des
    Füllgases die Atome ionisieren. Diese Ionen wirken in einem übersättigtem
    Wasserdampf entlang der Teilchenspur als Kondensationskeime für die Bildung
    von Wassertröpfchen, die durch Leichtstreuung sichtbar gemacht werden. Bis
    in die $70$'ger Jahre war die W.-Nebelkammer eine der wichtigsten
    Nachweisgeräte in Kern- und Teilchenphysik.
\end{enumerate}
\subsection{direkte Beobachtungen} % (fold)
\label{sub:direkte_Beobachtungen}
Mit einigen Instrumenten ist auch eine direkte Beobachtung von einzelnen Atomen
möglich:
\begin{enumerate}
    \item{Feldionenmikroskop.}
    Das ältere dieser Unstrumente ist das von E,Müller $1937$ entwickelte
    Feldionenmikroskop. Dabei dient eine kleine Wolframspitze als Kathode, der
    eine Anode in Form einer Halbkugel im Abstand $R \simeq 10-20cm$
    gegenübersteht. Bei eienr Spannung $U$ von $U \simeq 1kV$ ist das
    elektrische Feld $E = \frac{U}{r} \h{r}$ bei der Wolframspitze ($r \leq
    10nm$) groß genug ($ > 10^{111} \frac{V}{m}$), um Elektronen aus Materie zu
    reißen.

    Die Elektronen folgen den Feldlinien und treffen auf einen Leuchtschirm vor
    der anode auf. Die Vergrößerung ist hierbeu $\sim \frac{R}{r} \simeq 10^7$.
    Die Elektronen kommen dabei von Orten an denen die Austrittsarbeit minimal
    ist. Bringt man Atome mit kleinerer Austrittsarbeit auf die wolframspitze
    (z.B Barium), so kommen die $e^-$ überwiegend von diesen Atomen, die auf
    diese Weise mit Vergrößerung $\sim 10^7$ sichtbar gemacht werden.
    \item Das von E. Ruska ($1906$-$1988$, Nobelpreis $1986$) entwickelte
    Transmission-Elektronenmikroskop benutzt zur Ablenkung Elektronen, die auf hohe
    Spannungen ($\sim 500kV$) beschleunigt und durch geeignete $E$ und $B$
    Felder auf die  zu untersuchende (dünne) Probe fokusiert werden. Durch
    Stö߀ an der Probe werden $e^-$ abgelenkt, die transmittierten $e^-$ werden
    durch ein Ablenksystem auf einen Leuchtschirm abgebildet.
\end{enumerate}


% subsection direkte Beobachtungen (end)

% subsection indirekte Beobachtungen (end)

% section Kann man Atome sehen (end)
