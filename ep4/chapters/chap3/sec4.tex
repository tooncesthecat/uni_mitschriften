\section{Kann man Atome sehen?} % (fold)
\label{sec:Kann man Atome sehen}
Auch wenn Atome ($r=0.1 - 0.3 nm$) mit sichtbarem Licht ($\lambda = 500nm$)
nicht aufgelöst werden können, so sind eine reihe von Techniken und
Instrumenten entwickelt worden, die eine indirekte oder direkte Beobachtung von
Atomen erlauben.
\subsection{indirekte Beobachtungen} % (fold)
\label{sub:indirekte_Beobachtungen}
\begin{enumerate}
    \item Brownsche Bewegung
    \item Auch die von C. Wilson $1911$ entwickelte Nebelkammer erlaubt es,
    indirekt die Bahn einzelner mikroskopischer Teilchen wie Atome, Ionen,
    Elektronen, Positornen etc. sichtbar zu machen: Ein Teilchen mit genügend
    hoher kinetischer Energiea kann hierbei durch Stöse mit Atomen des
    Füllgases die Atome ionisieren. Diese Ionen wirken in einem übersättigtem
    Wasserdampf entlang der Teilchenspur als Kondensationskeime für die Bildung
    von Wassertröpfchen, die durch Leichtstreuung sichtbar gemacht werden. Bis
    in die $70$'ger Jahre war die W.-Nebelkammer eine der wichtigsten
    Nachweisgeräte in Kern- und Teilchenphysik.
\end{enumerate}
% subsection indirekte Beobachtungen (end)
\subsection{direkte Beobachtungen} % (fold)
\label{sub:direkte_Beobachtungen}
Mit einigen Instrumenten ist auch eine direkte Beobachtung von einzelnen Atomen
möglich:
\subsubsection{Feldionenmikroskop} % (fold)
\label{ssub:Feldionenmikroskop}
    Das ältere dieser Unstrumente ist das von E,Müller $1937$ entwickelte
    Feldionenmikroskop. Dabei dient eine kleine Wolframspitze als Kathode, der
    eine Anode in Form einer Halbkugel im Abstand $R \simeq 10-20cm$
    gegenübersteht. Bei eienr Spannung $U$ von $U \simeq 1kV$ ist das
    elektrische Feld $E = \frac{U}{r} \h{r}$ bei der Wolframspitze ($r \leq
    10nm$) groß genug ($ > 10^{111} \frac{V}{m}$), um Elektronen aus Materie zu
    reißen.

    Die Elektronen folgen den Feldlinien und treffen auf einen Leuchtschirm vor
    der anode auf. Die Vergrößerung ist hierbeu $\sim \frac{R}{r} \simeq 10^7$.
    Die Elektronen kommen dabei von Orten an denen die Austrittsarbeit minimal
    ist. Bringt man Atome mit kleinerer Austrittsarbeit auf die wolframspitze
    (z.B Barium), so kommen die $e^-$ überwiegend von diesen Atomen, die auf
    diese Weise mit Vergrößerung $\sim 10^7$ sichtbar gemacht werden.
% subsubsection Feldionenmikroskop (end)
\subsubsection{Transmission-Elektronenmikroskop} % (fold)
\label{ssub:Transmission-Elektronenmikroskop}
    Das von E. Ruska ($1906$-$1988$, Nobelpreis $1986$) entwickelte
    Transmission-Elektronenmikroskop benutzt zur Ablenkung Elektronen, die auf hohe
    Spannungen ($\sim 500kV$) beschleunigt und durch geeignete $E$ und $B$
    Felder auf die  zu untersuchende (dünne) Probe fokusiert werden. Durch
    Stöße an der Probe werden $e^-$ abgelenkt, die transmittierten $e^-$ werden
    durch ein Ablenksystem auf einen Leuchtschirm abgebildet.
% subsubsection Transmission-Elektronenmikroskop (end)
\subsubsection{Elektronenmikroskop} % (fold)
\label{ssub:Elektronenmikroskop}
% subsubsection Elektronenmikroskop (end)
\subsubsection{Rastertunnelmikroskop} % (fold)
\label{ssub:Rastertunnelmikroskop}
    Die größte Auflösung von Strutkturen auf leitenden Oberflachen erhält man
    mit dem $1984$ von G. Benning??? ($1947$-) und H. Roler??? ($1933-$)
    entwickelten Rastertunnelmikroskop (Nobelpreis $1986$). Dabei wird zwischen
    einer Wolframspitze (=Kathode) und einer metallischen Oberfläche (=Anode)
    eine kleine Spannung angelegt (einige $V$). Bei sehr kleinen Abständen zur
    Spitze u. Oberfläche können die $e^-$ aufgrund des sog. "Tunneleffektes"
    den kleinen Zwischenraum "durchtunneln". Der Tunnelstrom hängt dabei
    exponentiell vom Abstand ab. Hält man den Tunnelstrom konstant und fährt
    die Spitze über die Oberfläche, so bildet die Vertikalsbewegung der Spitze
    (die von einem Computer ausgelesen wird) die Topologie der Oberfläche ab.
% subsubsection Rastertunnelmikroskop (end)
\subsubsection{Atomares Kraftmikroskop (AF)} % (fold)
\label{ssub:Atomares_Kraftmikroskop_(AF)}
    Eine Weiterentwicklung des Tunnelmikropskops ist das sog. Atomare
    Kraftmikroskop (AF). Hier ist nicht der Tunnelstorm die Messgröße, sondern
    die atomaren Kräfte zwischen dem Oberflächenatomen und einer Spitze. In dem
    Fall kann das Mikroskop auch für nichtleitende Oberflächen verwendet
    werden. außerdem können Atome mit dem Mikroskop manipuliert werden.
% subsubsection Atomares Kraftmikroskop (AF) (end)
\subsubsection{Speichern und Beobachtung von einzelnen Atomen} % (fold)
\label{ssub:Speichern_und_Beobachtung_von_einzelnen_Atomen}
    Mit Hilfe von elektrischen und magnetischen DC-Feldern (Paulfalle, s III
    2.b) bzw
    elektrischen AC-Feldern (Paul-Falle, nach W.Paul ($1913-96$, Nobelpreis $1989$)
    gelang es ab den $60$ger Jahren zunächst Wolken von Ionen, und ab $1978$ auch
    einzelne Ionen zu speichern. Mit Hilfe von Laserkühlung können die Ionen
    bis nahe and den absoluten Nullpunkt der Tempertautr ($T < 1mK$) gekühlt
    werden. Durch Resonanzflouroszenz (Streuung des die Atome anregenden
    Laserlichts) können einzelne Atome sichtbar gemacht werden. Seit einigen
    Jahren gelingt es auch einzelne neutrale Atome mit Laserlicht zu speichern
    und sichtbar zu machen.
% subsubsection Speichern und Beobachtung von einzelnen Atomen (end)

% subsection direkte Beobachtungen (end)


% section Kann man Atome sehen (end)
