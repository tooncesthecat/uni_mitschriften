\section{Erzeugung von freien Elektronen} % (fold)
\label{sec:Erzeugung_von_freien_Elektronen}
Freie $e^-$ können auf verschiedene Weisen erzeugt werden. Die Wichtigsten
sind

\begin{enumerate}
    \item \textbf{Glühemission} \\
    Heizt man ein Metall (Kathode) auf eine hohe Temperatur $T$ auf, so können
    diejenigen von den frei beweglichen $e^-$ das  Metall verlassen, deren
    $E_{kin}$ größer ist als die Austrittsarbeit $W_A$.
    %BILD
    Für hohe Stromdichten verwendet man Materialien mit niederiger $W_A$, die
    hohe Temperaturen aushalten (z.B Wolfram-$Cr$-Legierung). Die Glühemission
    stellt die wichtigste Methode zur Erzeugung freier $e^-$ dar (Oszilloskop,
    Fernseher,Braunsche Röhre).
    \item \textbf{Feldemission} \\
    Beim Anlegen einer Spannung zwischen einer Kathode und Anode, z.B einer
    kleinen Spitze und einer flachen Anode, können an der Spitze sehr hoe
    Feldstärken erzeugt werden:
    \begin{equation*}
        \vec{E} = \frac{U}{r}  \h{r}
    \end{equation*}
    mit $r=10nm$ und $U=1kV$, ergeben sich Feldstärken $E > 10^{11}
    \frac{V}{m}$, so dass die $e^-$ aus dem Metall herausgerissen werden (s
    FEM).
    \item \textbf{Photoeffekt an Metalloberflächen} \\
    Beim Bestrahlen eines Metalls mit UV-Lichta können $e^-$ aus dem Metall
    austreten (s. V.3).
    \item \textbf{Sekundäremission aus Festköroperoberflächen} \\
    Beim Beschuss von Festkörperoberflächen mit $e^-$ können Sekundärelektronen
    aus dem Festkörper ausgelöst werden. Der Sekundäremissionskoeffizient $\eta$
    gibt an, wie viele Sekundärelektornen im Mittel pro einfallendem
    Primärelektron erzeugt werden ($\eta$ hängt von Material, Enfallswinkel und
    Energie des einfallendem $e^-$ ab).
    Die Sekundärelektron-Emission spielt eine wichtige Rolle in der
    Optoelektronik (Teilchendetektoren zum Nachweis von einzelnen $e^-$,
    Photonen, Ionen, sog. $e^-$-, Ionen- oder Photomultiplier, CCD,
    Bildverstärker. Für $m$ Dynoden erhält man pro Elektron $\eta^m$
    Elektronen, die an Kapazität $C_a$ die Spannung $U_a = \frac{\eta^m
    e^-}{C_a}$ erzeugen, die leicht nachgewiesen werden kann.
\end{enumerate}
% section Erzeugung von freien Elektronen (end)
