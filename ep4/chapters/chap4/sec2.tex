\section{Ladung des Elektrons} % (fold)
Eine genaue Abschätzung von $e$ erzielte $1913$ R. Millikan ($1868$-$1953$) mit
einem Experiment, bei dem er die Steig- bzw. Sinkgeschwindigkeit von geladenen
Öltröpfchen im elektrischen Feld eines Kondensators untersuchte. Dies ist auch
heute noch die genaueste Messung von $e$. Durch Zerstäuben von Öl werden
geladene Öltröpchen erzeugt die zwischen die Platten eines Kondensators sinken,
Durch Zerstäuben von Öl werden geladene Öltröpchen erzeugt die zwischen die
Platten eines Kondensators sinken, mit Ladung $Q=ne$,($n=1,2,3 \dots$). Im
feldfreien Raum wird die Schwerkraft $F_g = \rho \cd V \cd g$ durch die
Auftriebskraft $F_A = \rho_{Luft} \cd V \cd g$ und Stokesche Reibungskraft $F_R
= 6 \pi \eta R v$ kompensiert. Aus messung der Sinkgeschwindigkeit
\begin{equation*}
    R = \lk \frac{9 \eta v}{2g (\rho_{Öl} - \rho_{Luft})} \rk^{\frac{1}{2}}
\end{equation*}
und daraus das Volumen $V =\frac{4}{3}\pi R^3$.
Legy man eine Spannung $U$ an den Kondensatorplatten an, so können die
Öltröpfchen im elektrischen Feld $E=\frac{U}{d}$ zwischen den Platten in
Schwebe gehalten werden, falls $Q \cd E = n \cd e \cd E = \lk \rho_{Öl} -
\rho_{Luft} \rk \cd g \cd \frac{4}{3} \pi R^3$.
\begin{equation*}
    \rar n \cd e = \lk \rho_{Öl} - \rho_{Luft} \rk g \cd \frac{\frac{4}{3} \pi
    R^3}{E}
\end{equation*}
Zur Bestimmung der ganzen Zahl $n$ wird das Tröpchen im Kondensator durch
ionisierende Röntgenstrahlung umgeladen, so dass Ladungsänderung $
\Delta n \cd e$ auftreten, und die Spannung $U$ geändert werden muss um $\Delta
U = U_1 - U_2$, um das Tröpchen in Schwebe zu halten.
\begin{gather*}
    \rar \frac{n_2}{n_1} = \frac{n_1 + \Delta U}{n_1}=\frac{U_1}{U_2} \\
    \rar n_1 = \Delta n \frac{U_2}{U_1 -U_2}    
\end{gather*}
Für die kleinste Ladungsänderung ist $\Delta n =1$. Aus den entsprechenden
Werten $U_1$ und $U_2$ kann damit $n_1$ und daraus $e$ berechnet werden. Der
heute genaueste Wert ist:
\begin{equation*}
    \boxed{
    e = 1.60217653(13) \cd 10^{-19} C
    }
\end{equation*}
% section Ladung des Elektrons (end)
