\section{Die Masse des Elektrons} % (fold)
\label{sec:Die_Masse_des_Elektrons}
Alle Verfahren zur Bestimmung der Elektronenmasse beruhen auf die Ablenkung von
$e^-$ in einem elektrischen oder magnetischen Felnd. Hierbei wirkt die
Loretnzkraft:
\begin{gather*}
    \vec{F}=e \lk \vec{E} + \vec{v} \times \vec{B} \rk \\
    \ddot{\vec{r}} = \frac{e}{m_e}  \lk \vec{E} + \vec{v} \times \vec{B} \rk 
\end{gather*}
Man sieht, dass man aus dem Verlauf von $\vec{v}(t)$ immer nur $\frac{q}{m_e}$
erhalten kann, nie $e$ oder $m_e$ alleine. Zur Bestimmung von $m_e$ muss also
$e$ anderweitig ermittelt werden (z.B durch Millikian Versuch). zur Bestimmung
von $\frac{e}{m_e}$ kann dann z.B das Fadenstrahlrohr ($\sim \vec{B}$)
verwendet werden. Die Genauigkeit der Messung von $ \frac{e}{m_e}$ hat sich im
Luafe der Jahre ständig erhöht. Der Fehler in der Angabe von $m_e$ rührt
heutzutage haputsächlich in $e$ her.. Der heute aktuelle Wert lautet:
\begin{equation*}
    \boxed{
    m = 9.1093826(15) \cd 10^{-31}
    }
\end{equation*}


% section Die Masse des Elektrons (end)
