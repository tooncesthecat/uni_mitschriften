\section{Die Größe des Elektrons} % (fold)
\label{sec:Die_Größe_des_Elektrons}
Frage: 
\begin{quote}
    Welche konkrete Massen- und Ladungsverteilung besitzt dass $e^-$?
\end{quote}
Nimmt man vereinfachen an, dass das $e^-$ eine Kugel mit homogener Massen- und
Ladungsverteilung ist, kann man den "klassischen Elektronenradius $r_e$"
bestimmen.
%BILD
Ist Kapazität einer Vollkugel mit Radius $R$ ist
\begin{equation*}
    C = 2 \pi \epsilon_0 R
\end{equation*}
Die Arbeit zum Aufladen des Kondensators:
\begin{equation*}
    W = \frac{1}{22} \frac{Q^2}{C}
\end{equation*}
D.h für das Eleketron erhält man:
\begin{equation*}
    W = \frac{1}{2} \frac{e^2}{C} = \frac{e^3}{4 \pi \ep_0 r_e}
\end{equation*}
Setzt man $W = m_e c^2$, ergibt sich der "klassiche Elektronenradius" zu:
\begin{equation*}
    \boxed{
    r_e = \frac{e^2}{4 \pi \ep_0 m_e c^2} = 2.8 \cd 10^{-15} m
    }
\end{equation*}
Wie sich später im Zusammenhang mit dem Elektronenspin zeigen wird, führt
dieser Ansatz zu einer Reihe von Widersprüchen. Demnach ist dieses
klassisch-mechanische Modell des Elektrons falsch.

% section Die Größe des Elektrons (end)
