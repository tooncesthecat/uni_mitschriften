\section{Licht als klassiche Welle} % (fold)
\label{sec:Licht_als_klassiche_Welle}
Die Maxwell-Gleichungen führen auf die Wellengleichung und sagen die Existenz
von elektromagnetischen Wellen vorraus. Diese propagieren in Vakuum mit der
Geschwindigkeit $c$ in Richtung des Wellenvektors $\vec{k}$ und schwingen
transversal zu $\vec{k}$ mit Amplitude $E$ (bzw. $B= \frac{E}{2}$) entlang des
Polarisationsvektors $\vec{\ep}$.
\begin{equation*}
    \vec{E} = E \cd \vec{\ep} e^{i \lk \vec{k} \vec{r} - \omega t \rk }
\end{equation*}
Mit der Erzeugung von elektromagnetischen Wellen durch H.Hertz ($1857$-$1894$)
im Jahr $1887$ wurde die Maxwell-Theorie glänzend bestätigt. Darüber hinaus
wurde klar, dass Licht nur ein auf den Wellenlängebreich $\lambda = 0.4 - 0.7
\mu m$ beschränkter Spezialfall von elektromagnetischen Wellen ist.
\\
Die Energiedichte einer elektromagnetischer Welle ist
\begin{equation*}
    \mu = \frac{1}{2} \lk E^2 + c^2 B^2 \rk  = \ep_0 E^2
\end{equation*}
und mit Hilfe des Pointing-Vektors
\begin{gather*}
    \vec{S} = c^2 \ep_0 \vec{E} \times \vec{B} \\
    \lv \vec{S} \rv = c \ep_0 E^2 = c \mu
\end{gather*}
ist die Intensität definiert zu:
\begin{equation*}
    I = \left\langle \lv \vec{S} \rv \right\rangle
    = c \ep_0 \left\langle E^2 \right\rangle
    = c \left\langle \mu \right\rangle
\end{equation*}
Die drei Größen $\mu$,$\vec{S}$,$I$, sind hierbei kontinuierliche Funktionen von
$E$. Auch die Interferenzphänomene von Licht wie beugung oder Eigenschwinugen
in einem Resonator werden durch die Wellentheorie vollständig beschrieben.

% section Licht als klassiche Welle (end)
