\section{Die Hohlraumstrahlung} % (fold)
\label{sec:Die_Hohlraumstrahlung}
\subsection{Herleitung der Strahlungsleistung eines schwarzen Körpers durch
Planck} % (fold)
\label{sub:Herleitung_der_Strahlungsleistung_eines_schwarzen_Körpers_durch_Plan}
Heiße Körper senden aufgrund ihrer Temperatur elektromagnetische Strahlung aus.
Die Gesetze für die spektrale Intensitätsverteilung der Wärmestrahlung erhält
man aus der Analyse eines "schwarzen Strahlers", d.h. eines schwarzen Körpers,
dessen Absorptionsvermögen $A=1$ ist. Dieser wird annährend durch einen
Hohlraum mit absorbierenden Wänden realisiert, der eine kleine Öffnung hat:
Strahlung, die durch die Öffnung eintritt, wird i.d.R vollständig absorbiert,
bevor sie wieder austreten kann ($A\simeq1$).
Wenn man die Wände des Hohlraums auf eine Temperatur $T$ heizt, wirkt die
Öffnung als Strahlungsquelle. Im thermischen Gleichgewicht ist die aus dem
Raumwinkel $d\Omega$ im Frequenzintervall $d\nu$ von der Öffnung absorbierte
Strahlungsleistung gleich der nach $d\Omega$ innerhalb $d\nu$ von der Öffnung
emittierten Strahlungsleistung:
\begin{equation*}
    \frac{d W_A}{d t}= A_{\nu} S_{\nu} d\Omega d\nu 
    =
    E_{\nu} d\Omega d\nu = \frac{d W_e}{d t}
\end{equation*}
mit $S_{\nu} = S_{\nu} \lk \nu \Omega \rk = c \cd \mu_{\nu} \lk \nu \Omega \rk $
der spektralen Strahlungsdichte des Hohlraums. Mit $A_{\nu} =1 $ folgt $E_{\nu}
= S_{\nu}$. D.h. durch Messung der aus dem Licht austretenden Strahlungsleistung
$E_{\nu}$ lässt sich die Energiedichte $\mu_{\nu} = \frac{S_{\mu}}{c}$ im
Inneren des schwarzen Körpers bestimmen. 

In einem Hohlraum können dauerhaft nur Wellen oszillieren, bei denen ein
Vielfaches der halben Wellenlange gleich der Länge des Hohlraums ist
(Eigenmoden/Eigenschwingungen):
\begin{equation*}
    n_i =  \frac{\lambda}{L_i} \qquad (i = x,y,z)
\end{equation*}
bzw.
\begin{equation*}
    k_i = \frac{2\pi}{\lambda}=n_i \frac{\pi}{L}
\end{equation*}
\begin{beis}
    $n_x = 4$    
    %BILD
\end{beis}
Wie viele solcher Eigenmoden können in einem kubischen Resonator mit
Kantenlängen $L=L_x=L_y=L_y$ existierien? Im $\vec{k}$-Raum enstrpechen die erlaubten
Eigenmoden diskreten Punkten, die ein Gitter mit Gitterkonstanten
$\frac{\pi}{L}$ (im positiven Oktanden) bestehen. Pro Einheitsvolumen im
$\vec{k}$-Raum
befinden sich $\lk \frac{\pi}{L}\rk^{-3}$ erlaubter $k$-Werte, in einer Kugel
mit Radius $k_6$ also
\begin{equation*}
    \frac{1}{8}\frac{4}{3}\pi k_6^3 \lk \frac{L^3}{\pi^3} \rk 
    = \frac{1}{6}k_6^3 \frac{V}{\pi^2}
\end{equation*}
erlaubter $k$-Werte. Da jede Mode zwei unabhangige Polarisationsrichtungen hat,
befinden sich in dieser Kugel also insgesamt $2 \cd \frac{1}{6} k_6^3
\frac{V}{\pi^2}= \frac{k_6^3}{3}\frac{V}{\pi^2}$ solcher Moden. die Zahl der
moden pro Volumeneinheit $k<k_6$ ist also
\begin{equation*}
    n= \frac{N}{V}= \frac{k_6^3}{3\pi^2}=\frac{8\pi\nu^3}{3c^3}
\end{equation*}
d.h. im Frequenzintervall $d\nu$ befinden sich folgende Anzahl von Moden:
    \begin{equation*}
    \boxed{
        n_{\nu} d\nu 
        =
        \frac{d n}{d \nu} d\nu = \frac{8 \pi \nu^2}{c^3} d\nu
        }
    \end{equation*}
Die spektrale Enegiedichte $\mu_{\nu}$ ergibt sich daraus zu:
\begin{equation*}
    \mu_{\nu} d \nu = n_{\nu} \bar{W_{\nu}(T)} d\nu
\end{equation*}
%ORDENTLICHE BARS
mit $\bar{W_{\nu}(T)}$ der von der Temperatur abhängigen mittleren Energie pro
Mode im Frequenzintervall $\left[ \nu, \nu+d\nu\right]$. Rayleigh und James
ordneten $1894$ jeder Eigenschwingung wie beim klassichen harminischen
Oszillator) die mittlere Energie $k_B T$ zu.
\begin{equation*}
\tag*{\text{(Rayleigh-Jeansches Strahlungsgesetz)}}
\boxed{
    \mu_{\nu} d\nu = \frac{8\pi\nu^2}{c^2}k_B T d\nu
    }
\end{equation*}
Aus dem Loch des Hohlraums würde also die Strahlungsleistung:
\begin{equation*}
    S_{\nu} d\nu d\Omega 
    =
    c \mu_{\nu} d\nu \frac{d\Omega}{4\pi}
    =
    c \frac{1}{4\pi} \frac{8\pi\nu^3}{c^3}k_B T d\nu d\Omega
\end{equation*}
also
\begin{equation*}
    \boxed{
    S_{\nu} d\nu d\Omega 
    =
    \frac{2\nu^2}{c^2} k_B T d\nu d\Omega
    }
\end{equation*}
emittiert werden. Für niederige Frequenzen ergibt sich gute Übereinstimmung mi
den experimentellen Ergebnissen, für hohe Frequenzen treten große Diskrepanzen
auf, für $v\rar \infty$ divergiert $S_{\nu}$ sogar (Ultraviolett-Katastrophe).

Max Planck ($1858$-$1947$) lößte dieses Problem indem er $1900$ annahm, dass
jede Eigenmode Energie nicht in beliebig kleinen Beträgen besitzt, sondern nur
in diskreten Werten ("Energiequanten"). Diese hängen von der Grequenz der
Eigenmode ab und sind immer ganzzahlige Vielfache von $h \cd \nu$, dem
kleinstmöglichen Energiequant (=Photon). Die Energie einer eigenschwingung mit
$n$ Photonen der Frequenz $\nu$ ist demnach:
\begin{equation*}
    \tag*{(Planck'sches Wirkungsquantum)}
    \boxed{
    E_n = n \cd h \nu
    }
    \quad
    \text{mit}
    \quad
    \boxed{
        h=6.625 \cd 10^{-34} Js
    }
\end{equation*}
Im thermischen Gleichgewicht ist die Wahrschenilichkeit $P(E)$, dass eine
Eigenmode die Energie $E_n$ besitzt, durch den Boltzmanfaktor
$e^{-\frac{E_n}{k_B T}}$ gegeben, die normierte Wahrscheinlichkeit also durch
\begin{equation*}
    P(E_n)
    =
    \frac{e^{-\frac{n h \nu}{k_B T}}}{\sum_{n=0}^{\infty}e^{-\frac{n h \nu}{k_B T}}}
\end{equation*}
(mit $\sum p(E_n) =1$)
Die mittlere Energie pro Eigenmode ergibt sich damit zu
\begin{equation*}
    \bar{E}_n
    =
    \sum_{n=0}^{\infty} E_n p(E_n)
    =
    \frac{\sum n k \nu e^{-\frac{nh\nu}{k_B Ta}}}{\sum e^{-\frac{nh\nu}{k_B Ta}}}
\end{equation*}
\begin{equation*}
    E_n = \frac{1}{e^{-\frac{nh\nu}{k_B Ta}}-1}
\end{equation*}
\begin{erl}{Beweis}
(mit $\beta = \frac{1}{k_B T}$)
\begin{enumerate}
    \item 
    \begin{equation*}
        \sum_{n=0}^\infty e^{-n h \nu \beta}
        =
        \frac{1}{1- e^{-h\nu\beta}}\qquad \text{(mit $\sum_{n=0}^N x^n =
        \frac{1-x^{N+1}}{1-x}$)}
    \end{equation*}
    \item
    \begin{align*}
        \sum_0^{\infty} n h \nu e^{-n h \nu \beta}
        &=
        - \frac{\p}{\p \beta} \lk \sum_{n=0}^{\infty} e^{-n h \nu \beta} \rk
        \\
        &=
        - \frac{\p}{\p \beta} \lk \frac{1}{1-e^{-h \nu \beta}} \rk \\
        &=
        \frac{h \mu e^{-h \nu \beta}}{\lk 1 - e^{-h \nu \beta} \rk^2}
    \end{align*}
\end{enumerate}
also
\begin{equation*}
    \bar{E}_n = \frac{(2)}{(1)}= \frac{h\nu}{e^{\frac{h \nu}{k_B T}} -1}
\end{equation*}
\end{erl}
Daraus folgt die spektrale Energiedichte dr Hohlraumstrahlung:
\begin{equation*}
    \tag*{\text{(Plansch'sches Strahlungsgesetz)}}
    \boxed{ 
    \mu_{\nu} (\nu) d\nu =
    \frac{8 \pi \nu^2}{c^3} \frac{h \nu}{e^{\frac{h \nu}{k_B T}}-1} d\nu
    }
\end{equation*}
Die aus dem Loch des Hohlraums emittierte Strahlungsleistung ergibt sich daraus
zu:
\begin{gather*}
    S_{\nu} d\nu d\Omega = c \cd \frac{d\Omega}{4\pi} \frac{8 \pi h \nu^3}{c^3}
    \frac{d\nu}{e^{\frac{h\nu}{k_B T} -1}} \\
    \rar \boxed{
        S_{\nu} d\nu d\Omega 
        =
        \frac{2 h \nu^3}{c^2} \frac{d\nu d\Omega}{e^{\frac{h\nu}{k_B T}}-1}
        }
\end{gather*}
in sehr guter Übereinstimmung mit dem Experiment
\begin{erl}{Bemerkung}
Die gesamte Energiedichte strahlung ist
\begin{align*}
    \mu(T)
    &=
    \int_{v=0}^{\infty} \mu_{\nu} (v,T) d\nu
    =
    \frac{8 \pi h}{c^3} \int \frac{\nu^3}{e^{\frac{h \nu}{k_B T}}-1} d\nu \\
    &=
    \frac{8\pi^5 k_B^4}{15 h^3 c^3}\cdot T^4 = a \cd T^4  
\end{align*}
Die von dem Oberflächenelement in $d\Omega$ emittierte Strahlung ist also
\begin{equation*}
    S(T) = c \cd a \cd T^4 \frac{d\Omega}{4\pi} \cos \vartheta
\end{equation*}
d.h die Strahlungsleistung pro Flachenelement in dem gesamten Hohlraum ist
\begin{equation*}
   I = \int S(T) d\Omega = \sigma \cd T^4 \qquad \text{ mit $\sigma = \frac{2\pi^5
   k_B^4}{15c^2 h^3} = 5.67 \cd 10^{-8} \frac{1}{K^4}\frac{W}{m^2}$}
\end{equation*}
\end{erl}

% subsection Herleitung der Strahlungsleistung eines schwarzen Körpers durch Plan (end)
% section Die Hohlraumstrahlung (end)
