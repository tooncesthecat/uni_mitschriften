\section{Die Hohlraumstrahlung} % (fold)
\label{sec:Die_Hohlraumstrahlung}
\subsection{Herleitung der Strahlungsleistung eines schwarzen Körpers durch
Planck} % (fold)
\label{sub:Herleitung_der_Strahlungsleistung_eines_schwarzen_Körpers_durch_Plan}
Heiße Körper senden aufgrund ihrer Temperatur elektromagnetische Strahlung aus.
Die Gesetze für die spektrale Intensitätsverteilung der Wärmestrahlung erhält
man aus der Analyse eines "schwarzen Strahlers", d.h. eines schwarzen Körpers,
dessen Absorptionsvermögen $A=1$ ist. Dieser wird annährend durch einen
Hohlraum mit absorbierenden Wänden realisiert, der eine kleine Öffnung hat:
Strahlung, die durch die Öffnung eintritt, wird i.d.R vollständig absorbiert,
bevor sie wieder austreten kann ($A\simeq1$).
Wenn man die Wände des Hohlraums auf eine Temperatur $T$ heizt, wirkt die
Öffnung als Strahlungsquelle. Im thermischen Gleichgewicht ist die aus dem
Raumwinkel $d\Omega$ im Frequenzintervall $d\nu$ von der Öffnung absorbierte
Strahlungsleistung gleich der nach $d\Omega$ innerhalb $d\nu$ von der Öffnung
emittierten Strahlungsleistung:
\begin{equation*}
    \frac{d W_A}{d t}= A_{\nu} S_{\nu} d\Omega d\nu 
    =
    E_{\nu} d\Omega d\nu = \frac{d W_e}{d t}
\end{equation*}
mit $S_{\nu} = S_{\nu} \lk \nu \Omega \rk = c \cd \mu_{\nu} \lk \nu \Omega \rk $
der spektralen Strahlungsdichte des Hohlraums. Mit $A_{\nu} =1 $ folgt $E_{\nu}
= S_{\nu}$. D.h. durch Messung der aus dem Lich austretenden STrahlungsleistung
$E_{\nu}$ lässt sich die Energiedichte $\mu_{\nu} = \frac{S_{\mu}}{c}$ im
Inneren des schwarzen Körpers bestimmen.


% subsection Herleitung der Strahlungsleistung eines schwarzen Körpers durch Plan (end)
% section Die Hohlraumstrahlung (end)
