\section{Der photoelektrische Effekt} % (fold)
\label{sec:Der_photoelektrische_Effekt}
Bestrahlt man eine negativ aufgeladene Platte aus Metall mit UV-Licht, nimtt
die Ladung auf der Platte ab. Es müssen also $e^-$ die Platte verlassen.
\subsection{Klassische Erklärung} % (fold)
\label{sub:Klassische_Erklärung}
Im klassichen Wellenmodell sollte die auf die Fläche $F$ auftreffende
Strahlungsleistung $P=I\cd F$ gleichäßig auf alle $e^-$ verteilt werden. Bei
einer Eindringtiefe von $\sim \lambda$ im Metall und einer
Leitungs-$e^-$-Dichte von $n=\frac{10^{19}}{m^3}$ würde jedes $e^-$ im mittel
in der Zeit $\Delta t$ die Energie $\bar{\Delta W}$ aufnehmen:
\begin{equation*}
    \bar{\Delta W} = \frac{p}{n F \lambda} \cd \Delta t 
    =
    \frac{I}{n \lambda} \cd \Delta t
\end{equation*}
% subsection Klassische Erklärung (end)
und erst für den Fall $\bar{\Delta W} > W_A$ (=Austrittsarbeit) emittiert
werden.
\begin{beis}
    Zink: $W_a \sim 4eV$ und eine Quelle bi $\lambda \simeq 250nm$ mit $P=1W$
    im abstand von $1m$ von der zinkplatter egibt sich $\Delta \sim 10^5 s$ 
\end{beis}
Mit einer Gegenfeld-Methode kann man die maximale kinetische Energie der
emittierten $e^-$ $E_{kin}^{max}$ messen. die negative spannung an der anode
($-U_0$) bei der $I_{ph}$ einsetzt gibt an, gegen welche spannung die $e^-$
gerade noch anlaufen können. Dabei stellt sich heraus:
\begin{itemize}
    \item $E_{kin}^{max} = e U_{max}$ hängt nur von der Frequenz des UV-Lichts,
    nicht von seiner Intensität ab
    \item nur die Zahl der Photoelektronen, also $I_{ph}^{max}$, hängt von $I$
    ab
    \item
    trägt man $E_{kin}^{max}=e U_{max}$ gegen die Frequenz $\nu$ auf, ergibt
    sich eine Gerade
\end{itemize}
Einstein erklärte $1905$ die experimentellen Ergebnisse mit seiner
\begin{erl}{Lichtquanten-Hypothese}
    Jedes absorbierte Photon gibt Energie $h\nu$ vollständig an ein
    Photoelektron ab ("jeder Elementarprozess der Absorption und emission von
    Strahlung durch ein Atom erfolgt als 'gerichteter Vorgang'")
\end{erl}
Aus dem Energiesatz folgt:
\begin{equation*}
    E_{kin}^{max} = h\nu - W_A
\end{equation*}
mit $W_A$ der Austrittsarbeit des Kathodenmaterieals. Die Steigung der Geraden
ist also $h$ aus dem Achstenabschnitt bei $\nu = 0$ erhält man $W_A$.
\\
Die Unzulänglichkeit des Wellenmodells besteht also darin, dass in ihm die
Strahlungsenergie nicht auf ein einzelnes Atom oder Elektron übertragen wird.
Der Widerspruch, der sog. Welle-Teilchen-Dualismus, wird in Abschnitt $V 5$
noch genauer behandelt.
\subsection{Photonenimpuls und Laserkühlung} % (fold)
\label{sub:Photonenimpuls und Laserkühlung}
In dem Artikel von $1917$ forderte Einstein, dass be dem Elementarprozess der
Absorption und Emission von Strahlung durch die Atome nicht nur als gerichtete
Größe die Energe $h \nu$, sondern auch ein Impuls auf die Atome übertragen
wird. Dieser Impuls, der relativistisch dem Betrag $p = \frac{E}{c}$
entspricht, wird such auf die äußeren Freiheitsgrade der Atome auswirken.
\begin{erl}{Bemerkung:}
    relativistisch gilt: 
    \begin{equation*}
        E = \sqrt{p^2 c^2 + m_0^2 c^4}
    \end{equation*}
    mit $m_0=0$ $\rar$ $p=\frac{E}{c}$
    \begin{gather*}
        \rar p = \frac{E}{c}= \frac{h\nu}{c} = \frac{h}{\lambda} \\
        \rar \vec{p} = \hb \vec{k}
    \end{gather*}
    $\rar$ Lichtdruckkraft zur Laserkühlung!
\end{erl}

% subsection Photonenimpuls und Laserkühlung (end)

% section Der photoelektrische Effekt (end)
