\section{Compton Effekt} % (fold)
\label{sec:Compton_Effekt}
Der korpuskulare charakter der Lichtquanten tritt auch beim Compton-Effekt
($1922$) in Erscheinung (A.Compton $1892$-$1962$). bestrahlt man ein beliebiges
material mit Röntgenlicht der  Wellenlänge $\lambda_0$, so findet man in der
Streustrahlung außer $\lambda_0$ auch Anteile mit $\lambda_s > \lambda_0$. Im
Photonenmodell entspricht dies einem elastischen Stoß zwischen dem Photon der
energie $h\nu$ und Impuls $\hb \vec{k}$ und einem schwach gebundenen
(quasi-freien) Elektron des Streumaterials (das sich anfänglich in Ruhe
befindet):
\begin{equation*}
    h \nu_0 + e^- \longrightarrow h\nu_s + e^-
\end{equation*}
beim elastischen Stoß müssen Energie und Impuls erhalten bleiben, d.h:
\begin{equation*}
    \tag*{\text{(1)}}
    h \nu_0 + m_0 c^2 = h \nu_s + \gamma m_0 c^2
\end{equation*}
\begin{equation*}
    \tag*{\text{(2)}}
    \hb \vec{k}_0 = \hb \vec{k}_s + \vec{p}_e
\end{equation*}
mit $\gamma = \frac{1}{\sqrt{1 - \beta^2}} = \frac{1}{\sqrt{1 -
\frac{v^2}{c^2}}}$,$p_e= \gamma  m_0 \vec{v}$
\\
Aus $(1)$ ergibt sich durch Quadrieren und Umformen:
\begin{align*}
    \frac{m_0 v^2}{1 - \beta^2} = \frac{h^2}{c^2} \lk \nu_0 - \nu_s \rk + 2m_0 h
    \lk v_0 - v_s \rk \\
    \frac{m_0 v^2}{1 - \beta^2} = \frac{h^2}{c^2} \lk \nu_s^2 + \nu_0^2 - 2 \nu_s
    \nu_0 \cos \phi
\end{align*}
mit $\phi$ dem Winkel zur einfalls- und Streurichtung des Photons. Gleichsetzen
von $(3)$ und $(4)$ und umformen ergibt:
\begin{equation*}
    \nu_0 - \nu_s = \frac{h}{\m_0} \frac{\nu_0 \nu_s}{c^2} \lk 1-\cos \phi \rk 
\end{equation*}
Mit $(1-\cos \phi) = 2 \sin^2 (\frac{\phi}{2})$ und $\nu = \frac{c}{\lambda}$:
\begin{equation*}
    \tag*{\text{(Compton-Streu-Formel)}
    \boxed{
        \lambda_s = \lambda_0 + 2 \lambda_c \sin^2(\frac{\phi}{2})
    }
\end{equation*}
mit
\begin{equation*}
    \teg*{\text{(Compton Wellenlänge)}}
    \boxed{
        \lambda_C = \frac{h}{\m_0 c}= 2.4 \cd 10^{-12}m
    }
\end{equation*}
\begin{erlit}{Bemerkungen:}
    \item
    $\Delta \lambda = \lambda_s - \lambda_0$ ist unabhängig von der Primärwelle
    \item
    vom STreumaterial hängt nur die Intensität der Comptonstreung ab
    \item
    $\lambda_C$ entspricht der Wellenlänge einer Strahlung, deren Photonenergie
    der Ruheenergie des $e^-$ enstrpicht
    \begin{equation*}
        h \nu_C = \frac{hc}{\lambda_C} = m_0 c^2 = 511keV
    \end{equation*}
\end{erlit}
% section Compton Effekt (end)
