\section{Was ist Licht: Welle oder Teilchen?} % (fold)
\label{sec:Was_ist_Licht:_Welle_oder_Teilchen?}
Die in den letzten $3$ Abschnitten diskutierten Experimenten haben den
Teilchenaspekt elektromagnetischer Felder aufgezeigt: Jedes elektromagnetscihe
Feld der Frequenz $\nu$ besteht aus Energiequanten $h\nu$, den Photonen. Wie
gezeigt, lässt sich jedem Photon:
\begin{itemize}
    \item die Energie $h \mu$
    \item den Impuls $\hb \vec{k}$
    \item die Geschwindigkeit $v=c$, und daher
    \item die Ruhemasse $m_0 =0$ (sonst wäre $m=\gamma m_0=\infty$)
    \item den Eigendrehimpuls (Spin): $\vec{s}_{ph} = \pm \hb \cd  \h{e}_{\hb}$
\end{itemize}
zuordnen.
Bei Absorption von Licht durch freie Atome wir beobachtet, dass ein
links-zirkular-polarisiertes ($\simga^+$) Photon das in $z$-Richtung
propagiert, die Drehimpulskomponente $J_z$ des Atoms um
\begin{equation*}
    \Delta J_z = + \hb$ 
\end{equation*}
verändert, ein reichts-zirjular-polarisiertesPhoton ($\sigma^-$) um
\begin{equation*}
    \Delta J_z = - \hb
\end{equation*}
Photon besitzten also einen Drehimpuls $\vec{s}_{ph}$ ($s_{ph} = \pm \hb$), den
Phtonenspin, der in Propagationsrichtung zeigt.
\\
die Frage ist: was beschreibt elektromagnetische Strahlung richtig, das Wellen-
oder das Teilchenbild? Wo werden beispielsweise im Young'schen
Doppelspaltexperiment die Photonen beobachtet, wenn im Fernfeld gilt: $I(z) = c
\epsilon_0 \lv E_1 + E_2 \rv^2 = 2 c \epsilon_0 E_1^2 \lk 1 + cos
\Delta\phi(z)$
\\
Antwort: Die Energie wir nicht gleichförmig über den gesamten Raum verteilt,
sondern in Form von einzelnen Energiepaketen, den Photonen. Im Mittel werden
die Auftreffpunkte der Photonen so verteilt sein, dass sie insgesamt die
Interferenzstruktur des Young'schen Doppelspaltexperiments ergibt. Die
klassische Beschreibung der Elektromagnetischen Wellen bzw der Intensität
stellt also nur den Grenzfal großer Photonenzahlen dar und beschreibt nicht den
Einzelprozess. Für den Einzelprozess müssen elektromagnetische Wellen bzw. die
Intensität $I \sim \lv E \rb^2$ reinterpretiert werden als Funktionen, die die
statistische Aufenthaltswahrscheinlichkeit des einzelnen Photons beschreiben.
Insofern besitzt Licht sowohl Wellen- wie Teilchencharakter $\rar$
Welle-Teilchen-Dualismus.
% section Was ist Licht: Welle oder Teilchen? (end)
