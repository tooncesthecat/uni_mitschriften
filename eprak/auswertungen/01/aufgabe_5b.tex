\begin{frame}
\frametitle{Aufgabe 5b}
\framesubtitle{}
    \begin{itemize}
        \item Messung der Störung kleiner Spannungen durch Versuchsgeräte.
        \item Wiederholte Messung bei nähergelegten Kabeln
    \end{itemize}
\end{frame}
\begin{frame}
\frametitle{Aufgabe 5b}
\framesubtitle{}
\begin{center}
    \begin{tabular}{c|c}
        Gerät & Frequenz/kHz \\
        \hline
        Funktionsgenerator & 44.4 , 57,1, 62.3, 82,4, 76.0 \\
        DMM & 57.1 , 81.1 \\
        Monitor& 47.7 , 55.8, 64.2
    \end{tabular}

Mit Koaxialkabel: Keine Störungen. \\
$\rightarrow$  Der räumliche Versuchsaufbau hat Auswirkung auf die Messung \\
$\rightarrow$  Zur exakter Messung Störquellen vom Messort entfernen, oder
Koaxialkabel verwenden \\
$\rightarrow$ Keine unnötigen Geräte betreieben
\end{center}
\end{frame}
