\section{Wiederholung klassische Mechanik} % (fold)
\label{sec:Wiederholung_klassische_Mechanik}
Massepunkt in $1D$: $x(t)$ \\
$m \cd x(t) = F = -\frac{dU}{dx}$ \quad Impuls $p(t) = m \dot{x}(t)$ \\
$x(0) , p(0)$ bekannt $\longrightarrow$ $x(t), p(t)$ bestimmt für alle $t >0$. \\
Klassische Mechanik: kausal, deterministisch \\
Quantenmechanik: kausal, nicht deterministisch
\begin{erl}{Lagrange-Formalismus}
    \begin{equation*}
        L=T-v= \frac{1}{2} m \dot{x}^2 - V(x)
    \end{equation*}    

    Euler-Lagrange-Gleichung:

    \begin{align*}
        \frac{d}{dt} \frac{\p L}{\p \dot{x}} - \frac{\p L}{\p x} &= 0 \\
        \frac{d}{dt} m \dot{x} - \frac{\p V}{\p x} &= 0 \Rightarrow mx = - \frac{\p V}{\p x}
    \end{align*}
\end{erl}
\begin{erl}{Hamilton-Formalismus}
    \begin{equation*}
        p = \frac{\p L}{\p\dot{x}} = m \dot{x}
    \end{equation*}
    Hamilton-Funktion:
    \begin{align*}
        H(x,p)
        &=
        \frac{\p L}{\p \dot{x}} \dot{x} - L = p \frac{p}{m} - 
            \left[ \underbrace{\frac{1}{2} m \dot{x}^2}_{\frac{p^2}{2m}} - V(x)
            \right] \\
        &=
        \frac{1}{2m} p^2 + V(x) = T + V = E_{tot}
    \end{align*}
    Bewegungsgleichung:
    \begin{align*}
        \dot{x} &= \frac{\p H}{\p p} = \frac{p}{m} \\
        \dot{p} &= - \frac{\p H}{\p x} = - \frac{\p
        V}{\p x}
    \end{align*}
\end{erl}

