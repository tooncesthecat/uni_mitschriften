\section{Grenzen der klassischen Physik - Quantenphänomene} % (fold)
\label{sec:Grenzen_der_klassischen_Physik_-_Quantenphänomene}
\subsubsection{Eigenschaften elektromagnetischer Strahlung} % (fold)
\label{ssub:Eigenschaften_elektromagnetischer_Strahlung}
\begin{enumerate}
    \item{Hohlraumstrahlung} 
        Frequenzverteilung nicht klassisch erklärbar
    \item{Photoeffekt}  
\end{enumerate}
Erklärung: Planck(1900), Einstein (1905)

Lichtquanten (Photone) mit Energie $E= h \nu$ ($h=6.626 \cd 10^{-34} Js$
plancksches Wirkungsquantum).
% subsubsection Eigenschaften elektromagnetischer Strahlung (end)
\subsubsection{Eigenschaften von Materie} % (fold)
\label{ssub:Eigenschaften_von_Materie}
\begin{enumerate}
    \item{direkte Linien in der Spektroskopie} 
        Atommodelle: Rutherford, Bohr
    \item{Bewegung von Teilchen zeight Welleneigenschaften} 
        Postulat de Broglie (1923)
        Materieteilchen lassen sich wie Wellen beschreiben 
        $\lambda = \frac{h}{p} = \frac{h}{mv}$
\end{enumerate}
\begin{bei}
    \item Ball ($m=1kg$) mit Geschwindigkeit $v = 10 \frac{m}{s}$
     \begin{equation*}
         \lambda = \frac{h}{mv} = \frac{6.626 \cd 10^{-34}Js}{10 kg \frac{m}{s}} 
         = 6.6 \cd 10^{-35} m
     \end{equation*}
    \item Elektron mit Energie $100eV (\simeq 1.6 \cd 10^{-17}J)$
    \begin{equation*}
        \lambda = 1.2 \cd 10^{-10}m = 1.2\angstrom
    \end{equation*}
\end{bei}
Experimente: Davison / Germer (1927): Beugung von Elektronen an Kristallen.
% subsubsection Eigenschaften von Materie (end)
% section Grenzen der klassischen Physik - Quantenphänomene (end)
