\section{Wellenfunktion, Schrödingergleichung und die Postulate der
Quantenmechanik} % (fold)
\label{sec:Wellenfunktion,_Schrödingergleichung_und_die_Postulate_der_Quantenmech}
\subsection{1. Postulat (Wellenfunktion, Zustand)} % (fold)
\label{ssub:1._Postulat_(Wellenfunktion,_Zustand)}
Der Zustand eines \qmn Systems wird durch eine Wellenfunktion
(Zustandsfunktion/Zustand) $\Psi$ beschrieben, die i.A von den Koordinaten
aller Teilchen un der Zeit abhängt.

Z.B ein Teilchen $\vec{r}=(x,y,z) \quad \Psi(x,y,z,t)= \Psi(\vec{x},t)$

Die Wahrscheinlichkeit das Teilchen (zur Zeit $t$) im Bereich $V \in
\mathbb{R}^3$ zu finden ist gegeben durch
\begin{equation*}
    \int_V d^3 r \lv \Psi(\vec{r},t)\rv^2
\end{equation*}

Für infinitesimale Volumen $dV$ ist $\lv \Psi(\vec{r},t)\rv^2 dV$ die
Wahrscheinlichkeit das Teilchen in $dV$ um $\vec{r}$ zu finden.
    
$\varphi(\vec{r},t) = \lv \Psi(\vec{r},t)\rv^2$ ist die Wahrscheinlichkeitsdichte.

Es gilt die Normierungsbedingung
\begin{equation*}
    \int_{\mathbb{R}^3} d^3 r \lv \Psi(\vec{r},t)\rv^2=1
\end{equation*}
falls $\Psi$ quadratintegrabel.
\begin{bem}
    \item 
    Systeme mit $n$ Teilchen $\vec{r}_j = (x_j),y_j,z_j) \quad j=1 \ldots
    n \rightarrow \Psi(\vec{r}_1, \ldots, \vec{r}_n,t)$

    $H_2$ Molekül (2 Elektronen, 2 Protonen): $\Psi(r_1,r_2,r_3,r_4,t)$ 
    \item
    i.A $\Psi(\vec{r},t) \in \mathbb{C}$
    \item
    $\Psi$ ist bist auf globalen Phasenfaktor eindeutig 
    \begin{align*}
        \tilde{\Psi}(\vec{r},t)=
        e^{ia} \Psi(\vec{r},t) \\
        \tilde{\varphi}=\lv \tilde{\Psi} \rv^2 = \lv \Psi \rv^2 = \varphi
    \end{align*}
    \item Mathe: $\Psi \in W$, $W$ Vektorraum (Hilbertraum)

    Es gilt Superpositionsprinzip: $\Psi_1, \Psi_2 \in W$ mögliche
    Systemzustände, dann auch $c_1 \Psi_1 + c_2 \Psi_2 \in W$, $c_1,c_2 \in
    \mathbb{C}$. 

    Betrachte Teilchen mit Masse $m$, Geschwindigkeit $v$:
    %Bild

    Mit
    $k=\frac{2\pi}{\lambda},\lambda=\frac{h}{mv},k=\frac{mv}{\frac{h}{2\pi}}$
    folgt
    \begin{equation*}
        e^{ikz} = e^{\frac{i}{\hbar}mvz}
    \end{equation*}
    Rechts:
    \begin{align*}
        \Psi(x) &= \Psi_1(x) + \Psi_2(x) \\
        \Psi_{1/2}(x) &= n e^{\frac{imv}{2\hbar L}\left(x \pm \frac{d}{2}\right)^2} \\
        &= n e^{\frac{imv}{2\hbar L}\left(x^2 + \frac{d}{r}\right)} \
        e^{\pm \frac{imvxd}{2\hbar L}}
    \end{align*}
    Intensität am Schirm $\sim \varphi(x)$:
    \begin{align*}
       \varphi(x) &= \lv \Psi(x) \rv^2 = \lv \Psi_1(x) + \Psi_2(x)\lv^2 \\
                &= \lv \Psi_1(x) \rv^2 + \lv \Psi_2(x) \rv^2 \
                    + 2\Re \left( \Psi_1(x) \Psi_2^*(x) \right) \\
                &= \lv n \rv^2  \left( 1 + 1 + 2\Re \left( e^{-2i \
                    \frac{mvxd}{2\hbar L}}\right) \right)\\
                &= 2 \lv m \rv^2 \left( 1 + \cos \left(\frac{mvd}{\hbar L}x \right) \right)
    \end{align*}
    \item \qme Wahrscheinlichtkeitsbeschreibung
    \begin{equation*}
        \varphi(\vec{r},t) = \lv \Psi(\vec{r},t)\rv^2
    \end{equation*}
    Aussage über eine Vielzahl von Messungen an identischen Systemen.

    $\longrightarrow$ Indeterminisms
\end{bem}
    
% subsection 1. Postulat (Wellenfkt, Zustand) (end)
Messbare Größen in der klassischen Mechanik: Funktion auf dem Phasenraum:
\begin{description}
    \item[Funktion] $A(x,p)$
    \item[Energie] $E_{kin} = \frac{p^2}{2m} \rar E=H(x,p)$
    \item[Drehimpuls] $\vec{L} = \vec{r} \times \vec{p}$
\end{description}

\subsection{2. Postulat} % (fold)
\label{ssub:2._Postulat}
Jeder messbaren Gröse (Observable) $A$ ist ein (linear, hermetischer) Operator
$\h{A}$ zugeordnet. Man erhält $\h{A}$ indem man im klassischen Ausdruck
$A(x,p)$ dier Ersetzung durchführt.
\begin{align*}
    \underset{Impulsoperator}{p_z \rar \h{p_x} = \frac{\hbar}{i}\frac{\p}{\p
    x}} && \underset{Ortsoperator}{x \rar \h{x} \coloneqq x} \\
\end{align*}
Analog $3D$
\begin{align*}
    p_z \rar \h{p_y} = \frac{\hbar}{i}\frac{\p}{\p y} &&  y \rar \h{y} \coloneqq y \\
    p_z \rar \h{p_z} = \frac{\hbar}{i}\frac{\p}{\p z} &&  z \rar \h{z} \coloneqq z 
\end{align*}
\begin{bem}
    \item hermetische Operatoren: Mathe lineare Abbildungen $l \cl W \rar W$.

    Notation in der \QM:
    \begin{align*}
        \angstrom \cl W \rar W \\
        \angstrom \Psi(\vec{r},t) = \varphi(\vec{r},t)
    \end{align*}
    \begin{bei}
        \item{Ortsoperator}
        \begin{align*}
            \h{x}\Psi(x,y,z) &= x \Psi(x,y,z)\\
            \h{y}\Psi(x,y,z) &= y \Psi(x,y,z)
        \end{align*}
        \item{Impulsoperator}
        \begin{equation*}
            \tag{Differentialoperator}
            \h{p_x} \Psi (x,y,z) = \frac{\hbar}{-i}\frac{\p}{\p x}\Psi (x,y,z)
        \end{equation*}
        \item{kinetische Energie:} 
            \begin{align*}
                \text{Klassisch: } & T= \frac{p^2}{2m} \\
                \text{\QM: } & \h{T} = T(\h{x},\h{p}) 
                = \frac{1}{2m}\h{p_x}^2 
                = \frac{1}{2m}\frac{\hbar^2}{i^2}\frac{\p^2}{\p x^2} \\
                \h{T}\Psi(x)&= -\frac{\hbar^2}{2m}\frac{\p^2}{\p x^2}\Psi
            \end{align*}
            \begin{align*}
                \text{$3D$: } T &= \frac{1}{2m}\lk p_x^2, + p_y^2 + p_z^2 \rk \\
                \h{T} &= \frac{1}{2m}\lk \h{p}_x^2 + \h{p}_y^2 + \h{p}_z^2\rk\\
                &= -\frac{\hbar^2}{2m}\lk \frac{\p^2}{\p x^2}+\frac{\p^2}{\p 
                y^2}+\frac{\p^2}{\p z^2} \rk = \frac{\hbar^2}{2m}\Delta
            \end{align*}
        \item{Energie: Hamiltonoperator}
            \begin{align*}
                \text{klassisch: Hamiltonfunktion: } &
                H(r,p)=\frac{p^2}{m}+V(x)\\
                \text{\QM: } &\h{H} = H (\h{x},\h{p_1})=
                                \frac{1}{2m}\h{p_x}^2 + V(\h{x}) \\
                    &= -\frac{\hbar}{2m}\frac{\p^2}{\p x^2}+V(x)
            \end{align*}
            \begin{equation*}
                \h{H}\Psi(x) = \frac{\hbar^2}{2m}\frac{\p^2}{\p x^2}+V(x)\Psi(x)
            \end{equation*}
            \begin{equation*}
                \text{$3D$: } \frac{1}{2m}\lk \h{p_x}^2 + \h{p_y}^2 + \h{p_z}^2
                \rk + V (\h{x},\h{y},\h{z}) = - \frac{\hbar^2}{2m} \Delta + V (x,y,z)
            \end{equation*}
    \end{bei}
\end{bem}
% subsection 2. Postulat (end)
% section Wellenfunktion, Schrödingergleichung und die Postulate der Quantenmech (end)
\subsection{3. Postulat (Mittelwert, Erwartungswert)} % (fold)
\label{ssub:3. Postulat_(Mittelwert,_Erwartungswert)}
Der Mittelwert (Erwartungswert) einer physikalischen Observablen $A$ für ein
System im Zustand $\Psi$ ist
\begin{equation*}
    \left\langle \h{A} \right\rangle(t) = \int d^3r \Psi^* (\vec{r},t) \h{A} \Psi(\vec{r},t)
\end{equation*}

\begin{bem}
    \item{Beispiel}: $\Psi(x) = \frac{1}{\sigma^2 \pi}^{\frac{1}{4}} \
            e^{\frac{-\lk x-x_0 \rk^2 }{2 \sigma}}$

    Normierung: $\int_{-\infty}^{\infty} dx \lv \Psi(x)\rv^2 = 1$

    \begin{erl}{Mitterlwert der Ortsmessung:}
    \begin{align*}
        \left\langle \h{x} \right\rangle
        &=
        \int_{-\infty}^{\infty} dx \Psi^*(x)
        \underbrace{\h{x} \Psi(x)}_{x\Psi(x)} 
        =
        \int_{-\infty}^{\infty} dx
        \overbrace{\lv \Psi(x)\rv^2}^{\varphi(x)}x \\
        \left\langle \h{x} \right\rangle
        &=
        \int_{-\infty}^{\infty} dx \underbrace{x}_{x=x-x_0} \frac{1}{\lk \sigma \pi \rk}^{\frac{1}{4}} e^{-\frac{\lk x - x_0 \rk^2}{\sigma^2}}
        = x_0
    \end{align*}
    \end{erl}
    \begin{erl}{Mittelwert bei Impulsmessung}
    \begin{align*}
        \left\langle \h{p} \right\rangle
        &=
        \int_{-\infty}^{\infty} dx \Psi^*(x)\h{p}(x)\Psi(x)\\
        &=
        \int_{-\infty}^{\infty} \Psi^* \frac{\hbar}{i}
        \frac{\p}{\p x}\Psi(x)
    \end{align*}    
    und mit
    \begin{align*}
       \frac{\p}{\p x}\Psi(x) 
       &=
       \frac{1}{{\lk \sigma^2 \pi
       \rk}^{\frac{1}{4}}} \lk - \frac{\lk x- x_0 \rk}{\sigma^2}\rk 
       e^{- \frac{\lk x - x_0 \rk^2}{2\sigma^2}}
    \end{align*}
    folgt
    \begin{align*}
        \left\langle \h{p} \right\rangle
        &=
        \frac{\hbar}{i} \frac{1}{{\lk \sigma^2 \pi \rk}^{\frac{1}{2}}}
        \lk - \frac{1}{\sigma^2}\rk  \int_{-\infty}^{\infty} dx \lk x - x_0 \rk 
        e^{\frac{- \lk x - x_0 \rk^2}{\sigma^2}} = 0
    \end{align*}
    \end{erl}
    \item
    $\left\langle \h{A} \right\rangle \in \mathbb{R}$ $\rar$ Operator $\h{A}$
    hermitisch.
\end{bem}
% subsection 3. Postulat (Mittelwert, Erwartungswert) (end)
\subsection{4. Postulat (Messwert, Wahrscheinlichkeit)} % (fold)
\label{ssub:4._Postulat_(Messwert,_Wahrscheinlichkeit)}
Jede einzelne Messung der Observablen $A$ kann nur einen der Eigenwerte $a_n$
des Operators $\h{A}$ liefern, definiert durch
\begin{equation*}
    \tag{Eigenwertgleichung}
    \h{A}\varphi_n(\vec{r}) = a_n \varphi_n(\vec{r})
\end{equation*}
$\varphi_n$ Eigenfunktion von $\h{A}$ zum Eigenwert $a_n$.

Die Wahrscheinlichkeit $W_n$ bei einer Messung der Observablen $A$ an einem
System im Zustand $\Psi$ den Eigenwert $a_n$ zu finden, ist gegeben durch
\begin{equation*}
    W_n = \lv \int d^3 r \varphi_n^*(\vec{r}) \Psi(\vec{r},t)\rv^2
\end{equation*}
Nach der Messung ist das System in dem Eigenzustand $\varphi_n$, der dem
gemessenem Eigenwert entspricht.
\begin{bem}
    \item 
    Annahme $\varphi_n$ normiert $\lk \int d^3 r \lv \varphi_n
    (\vec{r})\rv^2 =1 \rk$
    \item
    In vielen Fällen: Eigenwert $a_n$ diskret $\rar$ Messgröße $A$ quantisiert

    Beispiel: mögliche Energien des $H$-Atoms
    \begin{equation*}
        E_n = - \frac{R}{n^2} \qquad R \approx 13.6eV, n \in \mathbb{R}
    \end{equation*}
    \item{Reduktion der Wellenfunktion}
    \QM: System wird i.A durch Messung verändert
    \begin{equation*}
        \Psi \xrightarrow[\text{Messung von $A$ mit Ergebnis $a_n$}]{} \varphi_n
    \end{equation*}
    $\rar$ sofortige weitere Messung von $A$ liefert wieder $a_n$.

    Zustand des Systems nach Messung $\varphi_n$:
    \begin{equation*}
        W_n = \lv \int d^3 r \varphi_n^*(\vec{r}) \Psi(\vec{r}) \rv^2
            = \lv \underbrace{\int d^3 r \varphi_n^*(\vec{r}) 
                        \varphi_n(\vec{r})}_{\int d^3 r \lv \varphi_n
                        (\vec{r}) \rv^2 = 1} \rv^2= 1
    \end{equation*}
    \item
    $\lv \Psi(x)\rv^2$ Wahrscheinlichkeitsdichte für Messung des Ortes $x$.

    \begin{erl}{Verallgemeinerte Eigenfunktion des Operators:}
     \begin{align*}
        \h{x}\varphi(x) &= \xi \varphi(x) \\
        x \varphi(x) &= \xi \varphi(x) \qquad \lk x-\xi \rk  f(x) = 0 \quad \forall x \\
        \varphi_\xi (x) &= \delta \lk x - \xi \rk 
     \end{align*}
     \begin{align*}
         W_\xi 
         &=
         \lv \int dx \varphi_\xi^* (x) \Psi(x) \rv^2 \\
         &=
         \lv \int dx \delta \lk x - \xi \rk \Psi(x) \rv^2 
         = \lv \Psi(\xi)\rv^2
     \end{align*}
    \end{erl}    
    \item
    Postulate $3$ und $4$ sind nicht unabhängig ($P4 \rar P3$)
    \begin{align*}
        \left\langle \h{x} \right\rangle
        &=
        \int dx \varphi(x)x = \int dx \lv \Psi(x) \rv^2 \\
        &=
        \int dx \Psi^*(x) x \Psi(x) = \int dx \Psi^*(x) \h{x} \Psi(x) \\
        &=
        \left\langle \h{x} \right\rangle
    \end{align*}
\end{bem}
% subsection 4. Postulat (Messwert, Wahrscheinlichkeit) (end)
\subsection{5. Postulat (Schrödingergleichung)} % (fold)
\label{sub:5._Postulat_(Schrödingergleichung)}
Die Zeitentwicklung der Zustände in der \QM wird durch die (zeitabhängige)
Schrödingergleichung bestimmt
\begin{equation*}
    \boxed{
        i \hbar  \frac{\p }{\p t } \Psi(\vec{r},t) = \h{H}\Psi(\vec{r},t)
        }
\end{equation*}
Dabei ist $\h{H}$ der Hamiltonoperator des Systems
\begin{bem}
    \item
    hier: nichtrelativisitsche \QM ($v <<  c$)
    \item
    \begin{align*}
        \h{H} &= \lk - \frac{\hbar}{2m}\frac{\p^2}{\p x}^2 + V(x) \rk \Psi(x,t) \\
        i\hbar \frac{\p}{\p t} \Psi(x,t) 
        &=
        - \frac{\hbar^2}{2m} \frac{\p^2 \Psi}{\p x^2} + V(x)
        \Psi(x,t)
    \end{align*}    
    Partielle DGL:
    \begin{equation*}
        \Psi(x,t_0) \xrightarrow[Löss]{über ???} \Psi(x,t)
    \end{equation*}
    \item{Plausibilität der Schrödinger Gleichung}
    Exp: Materie hat Welleneigenschaften $\rar$ Wellengleichung

    Elektrodynamik: 
    
    \begin{align*}
    \text{ebene Welle:} &\qquad\Psi(x,t)=Ae^{i \lk kx - \omega t \rk } \\
    \text{Dispersionsrelation} &\qquad\omega = kc 
    \end{align*}
    \begin{equation*}
        \frac{\p^2}{\p t^2} \Psi= - \omega^2 \Psi
        = - c^2 k^2 \Psi - c^2 \frac{\p^2}{\p x^2} \Psi
    \end{equation*}
    Wellengleichung in \QM?
    \begin{align*}
        \Psi(x,t)&= A e^{i \lk kx - \omega \rk} \\
        \text{de Broglie:} \qquad \lambda &= \frac{k}{p}
        \rar k = \frac{2\pi}{\lambda}=\frac{p}{\frac{k}{2\pi}} = \frac{p}{\hbar} \\
        \rar p &= \hbar k
    \end{align*}
    $E= \hbar \omega$ freies Teilchen $E= \frac{p}{2m}$
    \begin{equation*}
        \Rar \omega 
        = \frac{E}{\hbar}
        = \frac{1}{2m} \frac{p^2}{\hbar} =
        \boxed{
            \frac{tk^2}{2m} = \omega(k)     
        }
    \end{equation*}
    \begin{align*}
        \frac{d^2}{d t^2 } \Psi 
        &=
        - k^2 \Psi = \frac{2m\omega}{\hbar}\Psi
        =-\frac{2m}{\hbar} \frac{1}{-i} \frac{d }{dt} \Psi \\
        \Rar i \hbar \frac{d \Psi}{dt} 
        &= - \frac{\hbar^2}{2m} \frac{\p^2 \Psi}{\p t^2}
        =\underbrace{\h{H}\Psi}_{\h{H}=T}
    \end{align*}
\end{bem}

% subsection 5. Postulat (Schrödingergleichung) (end)
