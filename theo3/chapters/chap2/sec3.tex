\section{Das freie Teilchen - Wellenpakete} % (fold)
\label{sec:Das_freie_Teilchen_-_Wellenpakete}
Teilchen mit Masse $m$, eindimensional ohne Kräfte.
\begin{erl}{Hamiltonoperator:}
     \begin{equation*}
         \h{H}=\frac{\h{p}^2}{2m}+\underset{=0}{V(\h{x})}=\frac{\h{p}^2}{2m}= -
         \frac{\hb}{2m}\frac{\p^2}{\p x}
     \end{equation*}
\end{erl}
\begin{erl}{Schrödingergleichung}
      \begin{equation*}
          i \hb \frac{\p}{\p t} \Psi(x,t) = \h{H} \Psi(x,t) =
          -\frac{\hb^2}{2m}\frac{\p^2 \Psi}{\p x^2}
      \end{equation*}
\end{erl}
Separationsansatz $\Psi(x,t) = \varphi(x) \chi(t)$
\begin{align*}
    i \hb \varphi \frac{d \chi}{d t} &= -\frac{\hb^2}{2m} \frac{d^2
    \varphi}{dx^2 } \\
    &=
    i \hb \underbrace{\frac{1}{\chi}\frac{d x}{d t}}_{t}
    =
    - \frac{\hb^2}{2m} \underbrace{\frac{1}{\varphi} \frac{d^2
    \varphi}{dx^2}}_{x}
    = const = E
\end{align*}
\begin{gather*}
     \frac{d \chi }{dt}=- \frac{i}{\hb} E \chi \rar \chi(t) = c
     e^{-\frac{i}{\hb}E t} \\
     \underbrace{-\frac{\hb^2}{2m}\frac{d^2}{dx^2}}_{\h{H}}\varphi(x) = E \varphi(x)
\end{gather*}
\begin{equation*}
    \tag*{\text{(stationäre Schrödingergleichung)}}
    \h{H} \varphi(x) = E \varphi(x)
\end{equation*}
$\rar$ Eigenwertgleichung für $\h{H})$; $E$ mögliche Energien.

Lößung: $\varphi(x) = a e^{\pm ikx}$
\begin{gather*}
    - \frac{\hb^2}{2m} \frac{d ^2}{d x} \varphi(x)
    =
    \frac{\hb^2 k^2}{2m} \varphi = E \varphi
    \rar \frac{\hb k^2}{2m} = E \qquad E \geq 0 \\
    k = \frac{1}{\hb} \sqrt{ 2m E} \geq 0
\end{gather*}
$\rar$ Energie nicht quantisiert.

Zurück zur Separation:
\begin{gather*}
    \Psi(x,t) = c e^{\pm ikx - \frac{i}{\hb} Et}\\
    \omega = \frac{E}{\hb} = \frac{\hb k^2}{2m} = \omega(k)
\end{gather*}
\begin{equation*}
    \tag*{\text{(Dispersionsrelation)}}
    \pm ikx - i\omega(k)t = ce
\end{equation*}
% section Das freie Teilchen - Wellenpakete (end)

